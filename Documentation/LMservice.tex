\ProvidesFile{LMservice.tex}[v1.0.0]
\primaryStart[TheLMIService]{The~\LMI{}~Service}
The \asCode{mpmLeapMotionInputService} application is an Input service,
generating a stream of information on the position and orientation of one or more hands.
The application responds to the standard Input service requests and can be used as a
standalone data generator, without the need for a client connection.\\

The \emph{configure} request request has no arguments and does nothing.\\

The \emph{restartStreams} request stops and then starts the output stream.\\

The \emph{startStreams} request request initiates listening to the Leap Motion controller.
Once started, the service will send groups of hand data via the output \yarp{} network
connection.\\

The \emph{stopStreams} request stops the Leap Motion controller listener, which stops the
output \yarp{} network connection.\\ 

Note that the application will exit if the \emph{Registry Service} is not running.\\

\insertAppParameters
\insertTagDescription{\LMI}
\insertInputServiceComment
\condPage{}
If the application is running from a terminal and has not been automatically started via
the `\asCode{go}' option, the following commands are available:
\begin{itemize}
\item\cmdItem{?}{display a list of the available commands}
\item\exSp\cmdItem{b}{start the output stream, sending hand data}
\item\exSp\cmdItem{c}{configure the service; this has no effect, as the service has no
configurable parameters}
\item\exSp\cmdItem{e}{stop the output stream}
\item\exSp\cmdItem{q}{quit the application}
\item\exSp\cmdItem{r}{restart the output stream}
\item\exSp\cmdItem{u}{reset the configuration so that it will be reprocessed when the
output stream is restarted}
\end{itemize}
\primaryEnd
\primaryStart[TheTFIService]{The~\TFI{}~Service}
The \asCode{mpmTwoFingersInputService} application is an Input service,
generating a stream of information on the position of the first detected finger of the
first two detected hands.
The application responds to the standard Input service requests and can be used as a
standalone data generator, without the need for a client connection.\\

The \emph{configure} request request has no arguments and does nothing.\\

The \emph{restartStreams} request stops and then starts the output stream.\\

The \emph{startStreams} request request initiates listening to the Leap Motion controller.
Once started, the service will send pairs of finger position data via the output \yarp{}
network connection.\\

The \emph{stopStreams} request stops the Leap Motion controller listener, which stops the
output \yarp{} network connection.\\ 

Note that the application will exit if the \emph{Registry Service} is not running.\\

\insertAppParameters
\insertTagDescription{\TFI}
\insertInputServiceComment
\condPage{}
If the application is running from a terminal and has not been automatically started via
the `\asCode{go}' option, the following commands are available:
\begin{itemize}
\item\cmdItem{?}{display a list of the available commands}
\item\exSp\cmdItem{b}{start the output stream, sending hand data}
\item\exSp\cmdItem{c}{configure the service; this has no effect, as the service has no
configurable parameters}
\item\exSp\cmdItem{e}{stop the output stream}
\item\exSp\cmdItem{q}{quit the application}
\item\exSp\cmdItem{r}{restart the output stream}
\item\exSp\cmdItem{u}{reset the configuration so that it will be reprocessed when the
output stream is restarted}
\end{itemize}
\primaryEnd{}
