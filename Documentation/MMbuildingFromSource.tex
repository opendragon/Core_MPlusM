\ProvidesFile{MMbuildingFromSource.tex}[v1.0.0]
\appendixStart[BuildingFromSource]{\textitcorr{Building~\mplusm{}~From~Source}}%
An \mplusm{} installation is normally simply installed according to the
\emph{\MMM{}~Installation} manual.
It is also possible to build a customized \mplusm{} from the source files, which can be
done if the \asCode{Developer} package has also been installed.
\secondaryStart{Prerequisites}
\tertiaryStart{General~prerequisites}
At a minimum, the source files from the \asCode{Developer} package must be present on the
system, along with a \compLang{C++} development environment.
Note that the source files are provided as \textbf{ZIP} files that contain a snapshot of
the \asCode{git} repositories where the source code is stored.
\tertiaryEnd{}
\tertiaryStart{Prerequisites~for~Macintosh~OS~X}
\begin{itemize}
\item Xcode is available from the Macintosh App Store and provides the \compLang{C++}
development environment for Macintosh OS X
\item The Xcode command--line tools must be installed, using the command
`\asCode{xcode-select~{-}{-}install}'
\item The \asCode{ccmake} and \asCode{cmake} command--line tools, which can be installed
using \asCode{MacPorts} or directly from the web--page
\companyReference{http://www.cmake.org/download}{http://www.cmake.org/download}
\item The \asCode{autoconf213} command--line tool, which can be installed using
\asCode{MacPorts}; note that this is only needed if SpiderMonkey is being built
\item The \asCode{Packages} GUI tool, from
\companyReference{http://s.sudre.free.fr/Software/Packages/about.html}{St\'ephane Sudre};
this is only needed if you wish to create installer packages after building \mplusm
\end{itemize}
\tertiaryEnd{}
\tertiaryStart{Prerequisites~for~Microsoft~Windows}
\begin{itemize}
\item Visual Studio is available from Microsoft and provides the \compLang{C++}
development environment for Windows; the version used during development is
`Microsoft Visual Studio Express 2012 for Windows Desktop', but newer versions should also
work
\item The \asCode{cmake} tool, which can be installed from the web--page
\companyReference{http://www.cmake.org/download}{http://www.cmake.org/download}
\item An `\asCode{unzip}' program
\end{itemize}
\tertiaryEnd{}
\secondaryEnd{}
\secondaryStart{Initial~Steps}
\begin{itemize}
\item Create a directory to hold the source code; we'll call it `\asCode{MPM}'
\item Copy the desired \textbf{ZIP} files from their installation location
(\asCode{/opt/M+M/src} on Macintosh~OS~X) to the new directory \asCode{MPM}
\item \asCode{unzip} the \textbf{ZIP} files in the directory \asCode{MPM}
\item Remove the \textbf{ZIP} files from the \asCode{MPM} directory
\end{itemize}
\secondaryEnd{}
\secondaryStart{(Optional)~Building~\textbf{ACE}}
\textbf{ACE} (The ADAPTIVE Communication Environment) provides the low--level networking
support used within \mplusm{}; the files needed to compile and link with ACE are installed
as part of the normal \mplusm{} installation procedure, so it's not necessary to rebuild
ACE; the source code is provided in case there is an issue with the headers or libraries.
\tertiaryStart{Building~on~Macintosh~OS~X}
Note that the file \asCode{ACE\fatUnderscore{}wrappers-6.2.6/ace/config.h} will select the
correct configuration file for Macintosh OS X or Windows, so it does not need to be
modified.
\begin{itemize}
\item On the command--line, move to the directory containing the \mplusm{} source code,
\asCode{MPM}
\item Execute the command `\asCode{cd~ACE\fatUnderscore{}wrappers2}'
\item Execute the command `\asCode{export~ACE\fatUnderscore{}ROOT=\$(pwd)}'
\item Execute the command `\asCode{make}'
\item If the build is successful, execute the command `\asCode{sudo~sh}' and the following
commands within the new subshell:
\begin{itemize}
\item \asCode{export~ACE\fatUnderscore{}ROOT=\$(pwd)}
\item \asCode{export~DYLD\fatUnderscore{}LIBRARY\fatUnderscore{}PATH=/opt/M+M/lib:%
\$DYLD\fatUnderscore{}LIBRARY\fatUnderscore{}PATH}
\item \asCode{make~install}
\item \asCode{exit}
\end{itemize}
\item Execute the command `\asCode{sudo~./after-install.sh}'; this removes some unneeded
files
\end{itemize}
\tertiaryEnd{}
\tertiaryStart{Building~on~Microsoft~Windows}
TBD
\tertiaryEnd{}
\secondaryEnd{}
\secondaryStart{(Optional)~Building~\textbf{\yarp}}
\textbf{\yarp} (Yet Another Robot Platform) provides the connection management support
used within \mplusm{}; the files needed to compile and link with \yarp{} are installed as
part of the normal \mplusm{} installation procedure, so it's not necessary to rebuild
\yarp{}; the source code is provided in case there is an issue with the headers or
libraries.
\tertiaryStart{Building~on~Macintosh~OS~X}
\begin{itemize}
\item On the command--line, move to the directory containing the \mplusm{} source code,
\asCode{MPM}
\item Execute the command `\asCode{cd~yarp-master}'
\item Execute the command `\asCode{mkdir~build}'
\item Execute the command `\asCode{cd~build}'
\item Execute the command `\asCode{unset~ACE\fatUnderscore{}ROOT}'; this is necessary to
ensure that the installed version of \textbf{ACE} is used
\item Execute the command `\asCode{ccmake~..}'; this will open a graphical interface to the
makefile builder. If this is the first time that the command is executed within a
directory, the text `\asCode{EMPTY CACHE}' will be displayed -- press `\textbf{c}' to do
the initial configuration; there may be some warnings for project developers, which can be
cleared by pressing `\textbf{e}'
\item Change the `\textbf{CMAKE\fatUnderscore{}INSTALL\fatUnderscore{}PREFIX}' to
\asCode{/opt/M+M}
\item Change any other options that you are interested in applying; it is recommended that
the option\\
`\textbf{ENABLE\fatUnderscore{}YARPRUN\fatUnderscore{}LOG}' should be enabled and the
option `\textbf{CREATE\fatUnderscore{}YMANAGER}' should be disabled
\item Press `\textbf{c}' to configure the makefiles; there may be some CMake warnings for
project developers, which can be cleared by pressing `\textbf{e}'
\item Press `\textbf{g}' to generate the makefiles; there may be some CMake warnings for
project developers, which can be cleared by pressing `\textbf{e}'
\item Execute the command `\asCode{cmake~.}'; again, there will be some project developer
warnings, which can be safely ignored
\item Execute the command `\asCode{make}'
\item Execute the command `\asCode{sudo~../after-build.sh}'; this corrects some linking
issues with \yarp{} caused by installing it in a non--default location
\item Execute the command `\asCode{sudo~make~install}'
\item If desired, you can verify that \yarp{} is correctly built be executing the command
`\asCode{yarp}'
\end{itemize}
\tertiaryEnd{}
\tertiaryStart{Building~on~Microsoft~Windows}
TBD
\tertiaryEnd{}
\secondaryEnd{}
\secondaryStart{(Optional, used with \textit{Channel~Manager})~Building~\textbf{OGDF}}
\textbf{OGDF} (Open Graph Drawing Framework) provides the tools to layout graphs for the
Channel~Manager GUI application; the files needed to compile and link with OGDF are
installed as part of the normal \mplusm{} installation procedure, so it's not necessary to
rebuild OGDF; the source code is provided in case there is an issue with the headers or
libraries.
\tertiaryStart{Building~on~Macintosh~OS~X}
\begin{itemize}
\item On the command--line, move to the directory containing the \mplusm{} source code,
\asCode{MPM}
\item Execute the command `\asCode{cd~OGDF}'
\item Execute the command `\asCode{make~cleanrelease}'
\item Execute the command `\asCode{make~release}'
\item Execute the command `\asCode{sudo~make~install}'
\end{itemize}
\tertiaryEnd{}
\tertiaryStart{Building~on~Microsoft~Windows}
TBD
\tertiaryEnd{}
\secondaryEnd{}
\secondaryStart{(Optional)~Building~\textbf{SpiderMonkey}}
\textbf{SpiderMonkey}, is the JavaScript engine used with the JavaScript service or
\mplusm{}; the files needed to compile and link with SpiderMonkey are installed as part of
the normal \mplusm{} installation procedure, so it's not necessary to rebuild
SpiderMonkey.
The source files are not provided with the \asCode{Developer} package, but can be obtained
via the instructions on the web--page
\companyReference{https://developer.mozilla.org/en-US/docs/Mozilla/Projects/SpiderMonkey/%
Getting\_SpiderMonkey\_source\_code}%
{SpiderMonkey source code}, as \mplusm{} does not use a modified version of SpiderMonkey
and does not depend on a particular version -- other than it must be at least SpiderMonkey
\textbf{31}.
\tertiaryStart{Building~on~Macintosh~OS~X}
\begin{itemize}
\item On the command--line, move to the directory containing the SpiderMonkey source code,
which will most likely be named `\textbf{gecko}'
\item Execute the command `\asCode{cd~js/src}'
\item Execute the command `\asCode{autconf213}'
\item Execute the command `\asCode{mkdir~\fatUnderscore{}obj\fatUnderscore}'
\item Execute the command `\asCode{cd~\fatUnderscore{}obj\fatUnderscore}'
\item Create the file `replace-links.sh' with the following content:\\
\codeBegin{}
\#!/bin/bash\\
if test \$\# != 0\\
then\\
\tS\tS{}if test -d \$1\\
\tS\tS{}then\\
\tS\tS\tS\tS{}for ff in \$1/* \$1/*/*\\
\tS\tS\tS\tS{}do\\
\tS\tS\tS\tS\tS\tS{}if test -h "\$ff"\\
\tS\tS\tS\tS\tS\tS{}then\\
\tS\tS\tS\tS\tS\tS\tS\tS{}repl=\$(readlink "\$ff")\\
\tS\tS\tS\tS\tS\tS\tS\tS{}rm "\$ff"\\
\tS\tS\tS\tS\tS\tS\tS\tS{}cp -a "\$repl" "\$ff"\\
\tS\tS\tS\tS\tS\tS{}fi\\
\tS\tS\tS\tS{}done\\
\tS\tS{}fi\\
fi\\
\codeEnd{}
\item Execute the command `\asCode{../configure {-}{-}prefix=/opt/M+M/gecko
{-}{-}with-system-icu\\
{-}{-}enable-release {-}{-}enable-more-deterministic}'
\item Execute the command `\asCode{make}'
\item Execute the command `\asCode{sudo~make~install}'
\item Execute the command `\asCode{sudo~./replace-links~/opt/M+M/gecko/include/mozjs-}';
this replaces symbolic links in the header directories with the original items, so that it
is safe to distribute the resulting directories
\end{itemize}
\tertiaryEnd{}
\tertiaryStart{Building~on~Microsoft~Windows}
TBD
\tertiaryEnd{}
\secondaryEnd{}
\secondaryStart{Building~Core~\mplusm}
\tertiaryStart{Building~on~Macintosh~OS~X}
\begin{itemize}
\item On the command--line, move to the directory containing the \mplusm{} source code,
\asCode{MPM}
\item Execute the command `\asCode{cd~Core\fatUnderscore{}MPlusM-master}'
\item Execute the command `\asCode{ccmake~.}'; this will open a graphical interface to the
makefile builder. If this is the first time that the command is executed within a
directory, the text `\asCode{EMPTY CACHE}' will be displayed -- press `\textbf{c}' to do
the initial configuration
\item Change the `\textbf{CMAKE\fatUnderscore{}BUILD\fatUnderscore{}TYPE}' to
\asCode{Release}
\item Change any other options that you are interested in applying; the default options
minimize the amount of text that is output when \mplusm{} command--line tools are
executed
\item Press `\textbf{c}' to configure the makefiles
\item Press `\textbf{g}' to generate the makefiles
\item Execute the command `\asCode{cmake~.}'
\item Execute the command `\asCode{make}'
\item To run the unit tests for \mplusm{}, do the following:
\begin{itemize}
\item In a separate terminal session, launch the \yarp{} server by executing the command
`\asCode{yarp~server}'
\item In the terminal session where \mplusm{} was built, execute the command
`\asCode{make~test}'
\end{itemize}
\item To install the freshly--built \mplusm{} files, execute the command
`\asCode{sudo~make~install}'
\item To confirm the normal operation of \mplusm{}, do the following:
\begin{itemize}
\item Execute the command `\asCode{mpmRegistryService~\&}'; this will launch the Registry
Service in the background and return the \textbf{PID} that can be used to halt it later
\item Execute the command `\asCode{mpmServiceLister}; this should display the properties
of the Registry Service
\item If you wish to halt the Registry Service, execute the command `\asCode{halt -s HUP}
\#\#\#\#', where ``\#\#\#\#'' is the \textbf{PID} that was returned when the Registry
Service was launched
\end{itemize}
\end{itemize}
\tertiaryEnd{}
\tertiaryStart{Building~on~Microsoft~Windows}
TBD
\tertiaryEnd{}
\secondaryEnd{}
\secondaryStart{Building~\textit{Channel~Manager}}
\textit{Channel~Manager} is not built from the command--line but, rather, from the IDE.
\tertiaryStart{Building~on~Macintosh~OS~X}
\begin{itemize}
\item In Finder, open the directory containing the \mplusm source code, \asCode{MPM}
\item Open the directory `Utilities\fatUnderscore{}ChannelManager-master' within
\asCode{MPM}
\item Open the directory `Builds/MacOSX'
\item Open the file `Channel Manager.xcodeproj' with Xcode
\item Within Xcode, select either the `Channel Manager Debug' or `Channel Manager Release'
Scheme
\item `Build' or `Run' the project; to create an executable that can be placed in the
`Applications' folder, do an `Archive' / `Export' / `Export as a Mac Application'
\end{itemize}
\tertiaryEnd{}
\tertiaryStart{Building~on~Microsoft~Windows}
TBD
\tertiaryEnd{}
\secondaryEnd{}
\appendixEnd{}
