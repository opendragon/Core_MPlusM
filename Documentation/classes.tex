\ProvidesFile{classes.tex}[v1.0.0]
\primaryStart[ClassesAndTheirInterfaces]{Classes~and~Their~Interfaces}
The \mplusm{} classes are organized into seven namespaces:
\begin{itemize}
\item \textbf{\mplusm{}::Common} classes that support all the \mplusm{} activities
\item \textbf{\mplusm{}::Example} classes that demonstrate features of \mplusm{}
\item \textbf{\mplusm{}::Parser} classes that are used by the
\requestsNameR{Service~Registry}{ServiceRegistry}{match} request of the
\serviceNameR[Service~Registry]{ServiceRegistry}
\item \textbf{\mplusm{}::Registry} classes that work with the internal database to manage
the \mplusm{} services
\item \textbf{\mplusm{}::RequestCounter} classes that support measuring the performance
of an \mplusm{} installation
\item \textbf{\mplusm{}::Test} classes that are used during unit--testing of the \mplusm{}
source code
\item \textbf{\mplusm{}::Utilities} classes that are used only with the \mplusm{}
utility programs
\end{itemize}
There are 38 classes or structures in the \textbf{\mplusm{}::Common} namespace, 27 in
\textbf{\mplusm{}::Example}, 8 in \textbf{\mplusm{}::Parser}, 14 in
\textbf{\mplusm{}::Registry}, 6 in \textbf{\mplusm{}::RequestCounter}, 20 in
\textbf{\mplusm{}::Test} and 3 in \textbf{\mplusm{}::Utilities}.
The following sections describe the most critical of the classes in the
\textbf{\mplusm{}::Common} namespace -- many of the other classes, in all the namespaces,
derive from these classes. 
\secondaryStart{BaseClient}
Instances of derived classes of \secondaryRef{BaseClient}{BaseClient} are responsible to
mediate \mplusm{} communication with corresponding instances of derived classes of
\secondaryRef{BaseService}{BaseService}.\\

The bulk of the mechanisms needed to perform this are provided by the
\secondaryRef{BaseClient}{BaseClient} class itself, and the derived classes need only
provide service--specific routines, such as \asCode{getOneRandomNumber()},
which is a method of the \asCode{M+M::Example::RandomNumberClient} class.\\

The following methods are available to code that uses
\secondaryRef{BaseClient}{BaseClient} classes:
\begin{itemize}
\item \textbf{\asCode{FindMatchingService}} This static function is provided for the use
of utility programs that need to identify active services
\item \textbf{\asCode{addAssociatedChannel}} Adapter applications use this method to
inform the \serviceNameR[Service~Registry]{ServiceRegistry} about their input and output
\yarp{} network connections
\item \textbf{\asCode{connectToService}} This method is use by clients to establish a
connection to the endpoint of their companion service
\item \textbf{\asCode{disconnectFromService}} This method is used by clients to release
the connection to their companion service
\item \textbf{\asCode{findService}} This method is used by clients to locate the channel
to be used for communication with their companion service
\item \textbf{\asCode{reconnectIfDisconnected}} This method is used by clients to
re--establish a dropped connection with their companion service
\item \textbf{\asCode{removeAssociatedChannels}} Adapter applications use this method to
inform the \serviceNameR[Service~Registry]{ServiceRegistry} that their input and output
\yarp{} network connections are no longer available
\item \textbf{\asCode{send}} This method is used by clients to send a request to their
companion service
\end{itemize}
\secondaryEnd{}
\secondaryStart{BaseContext}
Instances of derived classes of \secondaryRef{BaseService}{BaseService} that have \yarp{}
network connections with persistent state use instances of
derived classes of \secondaryRef{BaseContext}{BaseContext} to track the information that
is retained from each request for the following request.
The \secondaryRef{BaseContext}{BaseContext} class itself provides no functionality.
\secondaryEnd{}
\secondaryStart{BaseInputOutputService}
The \secondaryRef{BaseInputOutputService}{BaseInputOutputService} class is designed to
support easy access to multiple instances of the same service, each of which can be used
to provide an interface to an external device or to a simple flow--through operation.\\

Classes derived from \secondaryRef{BaseInputOutputService}{BaseInputOutputService} are
required to provide implementations of the following methods:
\begin{itemize}
\item \textbf{\asCode{configure}} This method is invoked when a
\requestsNameR{\inputOutput{}}{InputOutput}{configure} request is sent to the service; it
is responsible for setting attributes of the input and output streams of the service
\item \textbf{\asCode{restartStreams}} This method is invoked when a
\requestsNameR{\inputOutput{}}{InputOutput}{restartStreams} request is sent to the
service; it is responsible for stopping and then starting the input and output streams of
the service
\item \textbf{\asCode{setUpStreamDescriptions}} This method is invoked when the service is
started and it is responsible for providing a list of port names and protocols for use by
the current instance of the service
\item \textbf{\asCode{startStreams}} This method is invoked when a
\requestsNameR{\inputOutput{}}{InputOutput}{startStreams} request is sent to the service;
it is responsible for starting the input and output streams of the service
\item \textbf{\asCode{stopStreams}} This method is invoked when a
\requestsNameR{\inputOutput{}}{InputOutput}{stopStreams} request is sent to the service;
it is responsible for stopping the input and output streams of the service
\end{itemize}
The \asCode{configure}, \asCode{restartStreams}, \asCode{startStreams} and
\asCode{stopStreams} methods can also be invoked by code that instantiates the service.
			
			\begin{Large}\textbf{TBD--TBD--TBD}\end{Large}.

\secondaryEnd{}
\secondaryStart{BaseMatcher}

			\begin{Large}\textbf{TBD--TBD--TBD}\end{Large}.

\secondaryEnd{}
\secondaryStart{BaseRequestHandler}

			\begin{Large}\textbf{TBD--TBD--TBD}\end{Large}.

\secondaryEnd{}
\secondaryStart{BaseService}

			\begin{Large}\textbf{TBD--TBD--TBD}\end{Large}.

\secondaryEnd{}
\secondaryStart{Endpoint}

			\begin{Large}\textbf{TBD--TBD--TBD}\end{Large}.

\secondaryEnd{}
\secondaryStart{ServiceRequest}

			\begin{Large}\textbf{TBD--TBD--TBD}\end{Large}.

\secondaryEnd{}
\secondaryStart{ServiceResponse}

			\begin{Large}\textbf{TBD--TBD--TBD}\end{Large}.

\secondaryEnd{}
\primaryEnd{}
