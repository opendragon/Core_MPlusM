\ProvidesFile{LMmessageFormat.tex}[v1.0.0]
\primaryStart{The~Message~Format}

Each time that the Leap Motion controller reports that there is information available,
the \asCode{LeapMotionInputService} packages it into a single message
\openSq\asCode{Bottle}\closeSq, containing the following:
\begin{itemize}
\item A list of hands
\item A list of tools
\end{itemize}

For each valid hand, a dictionary \openSq\asCode{Property}\closeSq{} is added to the list
of hands, with the following properties:
\begin{itemize}
\item \textbf{id} -- a numeric value for the hand, to separate data for different hands
\item \textbf{palmposition} -- the three--dimensional coordinates \openSq{}X, Y,
Z\closeSq{} of the palm; the units are millimetres from the Leap Motion origin
\item \textbf{palmnormal} -- the surface normal of the palm
\item \textbf{palmvelocity} -- the relative change of the palm position from its previous
position, in millimetres / second
\item \textbf{direction} -- the direction in which the hand is pointing, as a unit
three--dimensional vector
\item \textbf{arm} -- the arm information
\item \textbf{wristposition} -- the three--dimensional coordinates of the wrist
\item \textbf{confidence} -- the degree of confidence (from \asCode{0} to \asCode{1}) with
the data
\item \textbf{side} -- `\asCode{left}', `\asCode{right}' or `\asCode{unknown}' (which
should never appear)
\item \textbf{fingers} -- a list of fingers
\end{itemize}

The arm information consists of a single--element list containing a dictionary.
Since \yarp{} does not allow a dictionary to contain another dictionary directly, a list
is used to hold the arm--specific information.
The arm information has the following properties:
\begin{itemize}
\item \textbf{direction} -- the direction in which the arm is pointing, as a unit
three--dimensional vector
\item \textbf{elbowposition} -- the three--dimensional coordinates of the elbow
\end{itemize}

For each valid finger, a dictionary is added to the list of fingers, with the following
properties:
\begin{itemize}
\item \textbf{id} -- a numeric value for the finger, to separate data for different
fingers
\item \textbf{type} -- either `\asCode{thumb}', `\asCode{index}', `\asCode{middle}',
`\asCode{ring}', `\asCode{pinky}' or `\asCode{unknown}' (which should never appear)
\item \textbf{tipposition} -- the three--dimensional coordinates of the tip of the finger
\item \textbf{tipvelocity} -- the relative change of the finger tip position from its
previous position
\item \textbf{direction} -- the direction in which the finger tip is pointing, as a unit
three--dimensional vector
\item \textbf{length} -- the length of the finger in millimetres
\item \textbf{extended} -- `\asCode{yes}' or `\asCode{no}' if the finger is extended or
not
\item \textbf{bones} -- a list of bones
\end{itemize}

For each valid bone, a dictionary is added to the list of bones, with the following
properties:
\begin{itemize}
\item \textbf{proximal} -- the three--dimensional coordinates of the end of the bone that
is closest to the wrist
\item \textbf{distal} -- the three--dimensional coordinates of the end of the bone that is
closest to the finger tip
\item \textbf{direction} -- the direction in which the bone is pointing, as a unit
three--dimensional vector
\item \textbf{length} -- the length of the bone in millimetres
\end{itemize}
Note that the same number of bones are described for each finger, even though the thumb
actually has one fewer bone -- the first ``bone'' of a thumb in the list has zero
length.\\

For each valid tool, a dictionary is added to the list of tools, with the following
properties:
\begin{itemize}
\item \textbf{id} -- a numeric value for the tool, to separate data for different tools
\item \textbf{tipposition} -- the three--dimensional coordinates of the tip of the tool
\item \textbf{tipvelocity} -- the relative change of the tool tip position from its
previous position
\item \textbf{direction} -- the direction in which the tool is pointing, as a unit
three--dimensional vector
\item \textbf{length} -- the length of the tool in millimetres
\end{itemize}
\primaryEnd{}
