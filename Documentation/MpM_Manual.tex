\ProvidesFile{MpM_Manual.tex}

\documentclass[letterpaper,titlepage,twoside]{report}

% Begin Preamble:

%\usepackage[T1]{fontenc}
%\usepackage{palatino}
\usepackage{times}
%\usepackage{bookman}
%\usepackage{newcent}

\usepackage{amsmath}
\usepackage{enumerate}
\usepackage{ifthen}
\usepackage{alltt}
\usepackage{calc}
\usepackage{shortvrb}
\usepackage{varioref}
\usepackage[dvips]{graphicx}
\usepackage[dvips,usenames]{color}
\usepackage{makeidx}
\usepackage{xspace}
\usepackage{fancyhdr}
\usepackage[section]{tocbibind}

% Control the code, depending on whether a hyper-linked PDF is being generated:
\newboolean{generatingHyperPDF}
\setboolean{generatingHyperPDF}{true}

% If the package 'hyperref' is disabled by commenting out the following lines,
% be sure to set the boolean 'generatingHyperPDF' to false.
\ifthenelse{\boolean{generatingHyperPDF}}%
 {\usepackage[dvips,
    colorlinks=true,
    linkcolor=webgreen,
    filecolor=webbrown,
    citecolor=webgreen,
    urlcolor=webblue,
    pdftitle={Movement And Meaning Middleware},
    pdfauthor={Norman Jaffe},
    pdfkeywords={movement,middleware,YARP,ACE,JUCE},
    pdfsubject={Middleware},
    bookmarks,
    raiselinks=true,
    plainpages=false,
    bookmarksopen=true,
    pdfstartview=Fit,
    pdfpagemode=UseOutlines]{hyperref}}
 {\newcommand{\hyperpage}[1]{#1}}

\usepackage{mysects}

% Adjust the paper edges:
\setlength{\parindent}{0em}
\setlength{\textwidth}{\paperwidth-144pt}% 2"
\setlength{\marginparsep}{0pt}
\setlength{\marginparwidth}{0pt}
\setlength{\evensidemargin}{-18pt}% 0.25"
\setlength{\oddsidemargin}{-18pt}% 0.25"

% Some colours for the web:
\definecolor{webgreen}{rgb}{0,0.5,0}
\definecolor{webbrown}{rgb}{0.6,0,0}
\definecolor{webblue}{rgb}{0,0,0.5}

% Set up the page layout:
\pagestyle{fancyplain}
\newcommand{\mymark}{}
\lhead[]{\fancyplain{}{\textsc{\mymark}}}
\chead[]{}
\rhead[\fancyplain{}{\textsc{\mymark}}]{}
\lfoot[Page \thepage]{\today}
\cfoot[Movement~and~Meaning~Middleware]{Movement~and~Meaning~Middleware}
\rfoot[\today]{Page \thepage}
\renewcommand{\headrulewidth}{0.5bp}
\pagenumbering{roman}

% Set the float behaviour:
\setcounter{bottomnumber}{2}
\setcounter{totalnumber}{4}

% Suppress the normal numbering of sections, et cetera:
\setcounter{secnumdepth}{-3}
\setcounter{tocdepth}{2}

% A couple of useful commands to handle italic-to-normal transitions:
\newcommand{\textitcorr}[1]{\textit{#1}\/}
\newcommand{\emphcorr}[1]{\emph{#1}\/}
\newcommand{\nothing}{\ }
\newcommand{\textbfcorr}[1]{\textbf{#1}\/}
\newcommand{\textsfcorr}[1]{\textsf{#1}\/}

\newcommand{\mplusm}{\textbfcorr{M+M}}
\newcommand{\yarp}{\begin{footnotesize}\textsfcorr{YARP}\end{footnotesize}}

% First argument: index category
% Second argument: index subcategory
% Third argument: alternate index subcategory
% Fourth argument: index name
% Fifth argument: index suffix 
\newcommand{\multiindex}[5]{%
  \ifthenelse{\equal{#2}{!-!-!}}%
    {\index{#1!#4#5}}%
    {\index{#1!#2!#4#5}}}

% First argument: label category
% Second argument: label subcategory
% Third argument: alternate label subcategory
% Fourth argument: label name
\newcommand{\multilabel}[4]{%
  \ifthenelse{\equal{#3}{!-!-!}}%
    {\label{#1:#4}}%
    {\label{#1:#3:#4}}}

% First argument: reference category
% Second argument: reference subcategory
% Third argument: alternate reference subcategory
% Fourth argument: reference name
\newcommand{\multiref}[4]{%
  \ifthenelse{\equal{#3}{!-!-!}}%
    {\ref{#1:#4}}%
    {\ref{#1:#3:#4}}}

% The net effect is as follows:
%  generatingHyperPDF
%   'D' \textitcorr{\color{webgreen}#3}  \hypertarget{hyper.#2.#3}{}\label{#2:#3}  {}
%   'E' {}  {}  {}
%	'M' {} \hypertarget{hyper.#2.#3}{}\label{#2:#3} {}
%   'R' \hyperlink{hyper.#2.#3}{\textitcorr{#3}}  \index{#2!#3}  \ref{#2:#3}
%   'S' \textitcorr{#3}  \index{#2!#3}  {}
%   'X' \hyperlink{hyper.#2.#3}{\textitcorr{#3}}  {}  {}
%  not generatingHyperPDF
%   'D' \textitcorr{\color{webgreen}#3}  \index{#2!#3|(textbf}\label{#2:#3}  {}
%   'E' {}  \index{#2!#3|)textbf}  {}
%   'M' {}  \index{#2!#3|(textbf}\label{#2:#3}  {}
%   'R' \textitcorr{#3}  \index{#2!#3}  \ref{#2:#3}
%   'S' \textitcorr{#3}  \index{#2!#3}  {}
%   'X' \textitcorr{#3}  {}  {}

%   D = Define the object (emphasize the index, create a label);
%   E = End of the object definition (close the index, no text);
%	M = Define the object (no visible text)
%   R = Refer to the object in the index (the default);
%   S = Reference to a standard object and
%   X = Don't add a reference for the object to the index (any letter except D or
%         R could be used, X is preferred for mnemonic value)

% First argument: hyperlink/index category
% Second argument: hyperlink/index subcategory
% Third argument: alternate hyperlink/index subcategory
% Fourth argument: hyperlink/index name
% Fifth argument: alternate hyperlink/index name
\ifthenelse{\boolean{generatingHyperPDF}}%
  {\newcommand{\entityNameD}[5]{%  command if generatingHyperPDF
    \hypertarget{hyper.#1.#4}{}%
	\ifthenelse{\equal{#5}{!-!-!}}%
	  {\textitcorr{\color{webgreen}#4}\multiindex{#1}{#2}{#3}{#4}{|(textbf}}%
	  {\textitcorr{\color{webgreen}#5}\multiindex{#1}{#2}{#3}{#5}{|(textbf}}%
    \multilabel{#1}{#2}{#3}{#4}}}
  {\newcommand{\entityNameD}[5]{%  command if not generatingHyperPDF
	\ifthenelse{\equal{#5}{!-!-!}}%
	  {\textitcorr{\color{webgreen}#4}\multiindex{#1}{#2}{#3}{#4}{|(textbf}}%
	  {\textitcorr{\color{webgreen}#5}\multiindex{#1}{#2}{#3}{#5}{|(textbf}}%
    \multilabel{#1}{#2}{#3}{#4}}}

% First argument: hyperlink/index category
% Second argument: hyperlink/index subcategory
% Third argument: alternate hyperlink/index subcategory
% Fourth argument: hyperlink/index name
% Fifth argument: alternate hyperlink/index name
\ifthenelse{\boolean{generatingHyperPDF}}%
  {\newcommand{\entityNameE}[5]{}}%
  {\newcommand{\entityNameE}[5]{%
    \ifthenelse{\equal{#1}{#4}}%
    {}% if first and fourth argument match
    {\ifthenelse{\equal{#5}{!-!-!}}%
      {\multiindex{#1}{#2}{#3}{#4}{|)textbf}}%
      {\multiindex{#1}{#2}{#3}{#5}{|)textbf}}}}}

% First argument: hyperlink/index category
% Second argument: hyperlink/index subcategory
% Third argument: alternate hyperlink/index subcategory
% Fourth argument: hyperlink/index name
% Fifth argument: alternate hyperlink/index name
\ifthenelse{\boolean{generatingHyperPDF}}%
  {\newcommand{\entityNameM}[5]{%  command if generatingHyperPDF
	\hypertarget{hyper.#1.#4}{}%
	\ifthenelse{\equal{#5}{!-!-!}}%
	  {\multiindex{#1}{#2}{#3}{#4}{|(textbf}}%
	  {\multiindex{#1}{#2}{#3}{#5}{|(textbf}}%
  	\multilabel{#1}{#2}{#3}{#4}}}
  {\newcommand{\entityNameM}[5]{%  command if not generatingHyperPDF
	\ifthenelse{\equal{#5}{!-!-!}}%
	  {\multiindex{#1}{#2}{#3}{#4}{|(textbf}}%
	  {\multiindex{#1}{#2}{#3}{#5}{|(textbf}}%
    \multilabel{#1}{#2}{#3}{#4}}}

% First argument: hyperlink/index category
% Second argument: hyperlink/index subcategory
% Third argument: alternate hyperlink/index subcategory
% Fourth argument: hyperlink/index name
% Fifth argument: alternate hyperlink/index name
\ifthenelse{\boolean{generatingHyperPDF}}%
  {\newcommand{\entityNameR}[5]{%  command if generatingHyperPDF
	\ifthenelse{\equal{#5}{!-!-!}}%
	  {\hyperlink{hyper.#1.#4}{\textitcorr{#4}}%
	  \ifthenelse{\equal{#1}{#4}}%
	    {}% if first and fourth argument match
	    {\multiindex{#1}{#2}{#3}{#4}{}}}%
	  {\hyperlink{hyper.#1.#4}{\textitcorr{#5}}%
	  \ifthenelse{\equal{#1}{#4}}%
	    {}% if first and fourth argument match
	    {\multiindex{#1}{#2}{#3}{#5}{}}}%
  	\multiref{#1}{#2}{#3}{#4}}}
  {\newcommand{\entityNameR}[5]{%  command if not generatingHyperPDF
  	\ifthenelse{\equal{#5}{!-!-!}}%
	  {\textitcorr{#4}%
	  \ifthenelse{\equal{#1}{#4}}%
	    {}% if first and fourth argument match
	    {\multiindex{#1}{#2}{#3}{#4}{}}}%
	  {\textitcorr{#5}%
	  \ifthenelse{\equal{#1}{#4}}%
	    {}% if first and fourth argument match
	    {\multiindex{#1}{#2}{#3}{#5}{}}}%
    \multiref{#1}{#2}{#3}{#4}}}

% First argument: hyperlink/index category
% Second argument: hyperlink/index subcategory
% Third argument: alternate hyperlink/index subcategory
% Fourth argument: hyperlink/index name
% Fifth argument: alternate hyperlink/index name
\newcommand{\entityNameS}[5]{%
  \ifthenelse{\equal{#5}{!-!-!}}%
	{\textitcorr{#4}}%
	{\textitcorr{#5}}%
  \ifthenelse{\equal{#1}{#4}}%
    {}% if first and fourth argument match
    {\multiindex{#1}{#2}{#3}{#4}{}}}

% First argument: hyperlink/index category
% Second argument: hyperlink/index subcategory
% Third argument: alternate hyperlink/index subcategory
% Fourth argument: hyperlink/index name
% Fifth argument: alternate hyperlink/index name
\ifthenelse{\boolean{generatingHyperPDF}}%
  {\newcommand{\entityNameX}[5]{%  command if generatingHyperPDF
	\ifthenelse{\equal{#5}{!-!-!}}%
	  {\hyperlink{hyper.#1.#4}{\textitcorr{#4}}}%
	  {\hyperlink{hyper.#1.#5}{\textitcorr{#5}}}}}
  {\newcommand{\entityNameX}[5]{%  command if not generatingHyperPDF
  \ifthenelse{\equal{#5}{!-!-!}}%
	{\textitcorr{#4}}%
	{\textitcorr{#5}}}}

% Use \entityReference, rather than \entityName, for the first mention of an object within
% another object, so that page ranges will be present.
\ifthenelse{\boolean{generatingHyperPDF}}%
  {\newcommand{\entityReference}[2]{\entityNameR{#1}{!-!-!}{!-!-!}{#2}{!-!-!}}}%  command if generatingHyperPDF
  {\newcommand{\entityReference}[2]{\entityNameR{#1}{!-!-!}{!-!-!}{#2}{!-!-!} \vpageref[(][(]{#1:#2})}}%  command if not generatingHyperPDF

\ifthenelse{\boolean{generatingHyperPDF}}%
  {\newcommand{\companyReference}[2]{\href{#1}{#2}}}%  command if generatingHyperPDF
  {\newcommand{\companyReference}[2]{#2}}% command if not generatingHyperPDF

% First argument [optional]: alternate name
% Second argument: entity name
\newcommand{\clientNameD}[2][!-!-!]{\entityNameD{Clients}{!-!-!}{!-!-!}{#2}{#1}}% shortcut
\newcommand{\clientNameE}[2][!-!-!]{\entityNameE{Clients}{!-!-!}{!-!-!}{#2}{#1}}% shortcut
\newcommand{\clientNameM}[2][!-!-!]{\entityNameM{Clients}{!-!-!}{!-!-!}{#2}{#1}}% shortcut
\newcommand{\clientNameR}[2][!-!-!]{\entityNameR{Clients}{!-!-!}{!-!-!}{#2}{#1}}% shortcut
\newcommand{\clientNameS}[2][!-!-!]{\entityNameS{Clients}{!-!-!}{!-!-!}{#2}{#1}}% shortcut
\newcommand{\clientNameX}[2][!-!-!]{\entityNameX{Clients}{!-!-!}{!-!-!}{#2}{#1}}% shortcut

% First argument [optional]: alternate name
% Second argument: entity name
\newcommand{\serviceNameD}[2][!-!-!]{\entityNameD{Services}{!-!-!}{!-!-!}{#2}{#1}}% shortcut
\newcommand{\serviceNameE}[2][!-!-!]{\entityNameE{Services}{!-!-!}{!-!-!}{#2}{#1}}% shortcut
\newcommand{\serviceNameM}[2][!-!-!]{\entityNameM{Services}{!-!-!}{!-!-!}{#2}{#1}}% shortcut
\newcommand{\serviceNameR}[2][!-!-!]{\entityNameR{Services}{!-!-!}{!-!-!}{#2}{#1}}% shortcut
\newcommand{\serviceNameS}[2][!-!-!]{\entityNameS{Services}{!-!-!}{!-!-!}{#2}{#1}}% shortcut
\newcommand{\serviceNameX}[2][!-!-!]{\entityNameX{Services}{!-!-!}{!-!-!}{#2}{#1}}% shortcut

\newcommand{\serviceReference}[1]{\entityReference{Services}{#1}}

% First argument [optional]: alternate name
% Second argument: entity name
\newcommand{\utilityNameD}[2][!-!-!]{\entityNameD{Utilities}{!-!-!}{!-!-!}{#2}{#1}}% shortcut
\newcommand{\utilityNameE}[2][!-!-!]{\entityNameE{Utilities}{!-!-!}{!-!-!}{#2}{#1}}% shortcut
\newcommand{\utilityNameM}[2][!-!-!]{\entityNameM{Utilities}{!-!-!}{!-!-!}{#2}{#1}}% shortcut
\newcommand{\utilityNameR}[2][!-!-!]{\entityNameR{Utilities}{!-!-!}{!-!-!}{#2}{#1}}% shortcut
\newcommand{\utilityNameS}[2][!-!-!]{\entityNameS{Utilities}{!-!-!}{!-!-!}{#2}{#1}}% shortcut
\newcommand{\utilityNameX}[2][!-!-!]{\entityNameX{Utilities}{!-!-!}{!-!-!}{#2}{#1}}% shortcut

% First argument [optional]: alternate name
% Second argument: Subcategory
% Third argument: entity name
\newcommand{\examplesNameD}[3][!-!-!]{\entityNameD{Examples}{#2}{#2}{#3}{#1}}% shortcut
\newcommand{\examplesNameE}[3][!-!-!]{\entityNameE{Examples}{#2}{#2}{#3}{#1}}% shortcut
\newcommand{\examplesNameM}[3][!-!-!]{\entityNameM{Examples}{#2}{#2}{#3}{#1}}% shortcut
\newcommand{\examplesNameR}[3][!-!-!]{\entityNameR{Examples}{#2}{#2}{#3}{#1}}% shortcut
\newcommand{\examplesNameS}[3][!-!-!]{\entityNameS{Examples}{#2}{#2}{#3}{#1}}% shortcut
\newcommand{\examplesNameX}[3][!-!-!]{\entityNameX{Examples}{#2}{#2}{#3}{#1}}% shortcut

% First argument [optional]: alternate name
% Second argument: Subcategory
% Third argument: Alternate subcategory name
% Fourth argument: entity name
\newcommand{\requestsNameD}[4][!-!-!]{\entityNameD{Requests}{#2}{#3}{#4}{#1}}% shortcut
\newcommand{\requestsNameE}[4][!-!-!]{\entityNameE{Requests}{#2}{#3}{#4}{#1}}% shortcut
\newcommand{\requestsNameM}[4][!-!-!]{\entityNameM{Requests}{#2}{#3}{#4}{#1}}% shortcut
\newcommand{\requestsNameR}[4][!-!-!]{\entityNameR{Requests}{#2}{#3}{#4}{#1}}% shortcut
\newcommand{\requestsNameS}[4][!-!-!]{\entityNameS{Requests}{#2}{#3}{#4}{#1}}% shortcut
\newcommand{\requestsNameX}[4][!-!-!]{\entityNameX{Requests}{#2}{#3}{#4}{#1}}% shortcut

\newcommand*{\insertpart}[2]{\clearpage\renewcommand{\mymark}{#1}#2}

% First argument [optional]: alternate hyperlink name
% Second argument: section title
% Third argument: hyperlink section
% Fourth argument: simplified version of title
% Fifth argument: prefix to display before title
\ifthenelse{\boolean{generatingHyperPDF}}%
  {\newcommand*{\sectionStart}[5][!-!-!]{\clearpage\section{#5\texorpdfstring{#2}{#4}}%
   \renewcommand{\mymark}{#4}%
   \ifthenelse{\equal{#1}{!-!-!}}%
    {\hypertarget{hyper.#3.#4}{}}%
    {\hypertarget{hyper.#3.#1}{}}}}%
  {\newcommand*{\sectionStart}[5][!-!-!]{\clearpage\section{#5#2}\renewcommand{\mymark}{#4}}}

\newcommand*{\sectionEnd}[1]{#1} % just a notational convenience

% First argument: [optional] alternate hypertarget name
% Second argument: subsection title
% Third argument: hypertarget section
% Fourth argument: simplified version of title
\ifthenelse{\boolean{generatingHyperPDF}}%
  {\newcommand*{\subsectionStart}[4][!-!-!]{\subsection{\texorpdfstring{#2}{#4}}%
   \ifthenelse{\equal{#1}{!-!-!}}%
    {\hypertarget{hyper.#3.#4}{}}%
    {\hypertarget{hyper.#3.#1}{}}}}%
  {\newcommand*{\subsectionStart}[4][!-!-!]{\subsection{#2}}}
    
\newcommand*{\subsectionEnd}[1]{#1} % just a notational convenience

% First argument: [optional] alternate hypertarget name
% Second argument: subsubsection title
% Third argument: hypertarget section
% Fourth argument: simplified version of title
\ifthenelse{\boolean{generatingHyperPDF}}%
  {\newcommand*{\subsubsectionStart}[4][!-!-!]{\subsubsection{\texorpdfstring{#2}{#4}}%
   \ifthenelse{\equal{#1}{!-!-!}}%
    {\hypertarget{hyper.#3.#4}{}}%
    {\hypertarget{hyper.#3.#1}{}}}}%
  {\newcommand*{\subsubsectionStart}[4][!-!-!]{\subsubsection{#2}\index{#2}}}
    
\newcommand*{\subsubsectionEnd}[1]{#1} % just a notational convenience

% First argument: [optional] alternate section name
% Second argument: section title
\newcommand*{\primaryStart}[2][!-!-!]{%
  \sectionStart[#1]{#2}{Primary}{#2}{}}

\newcommand*{\primaryEnd}[1]{#1} % just a notational convenience

% First argument: [optional] alternate subsection name
% Second argument: subsection title
\newcommand*{\secondaryStart}[2][!-!-!]{%
  \subsectionStart[#1]{#2}{Secondary}{#2}}

\newcommand*{\secondaryEnd}[1]{#1} % just a notational convenience

% First argument: [optional] alternate subsubsection name
% Second argument: subsubsection title
\newcommand*{\tertiaryStart}[2][!-!-!]{%
  \subsubsectionStart[#1]{#2}{Tertiary}{#2}}

\newcommand*{\tertiaryEnd}[1]{#1} % just a notational convenience

% First argument: [optional] alternate appendix name
% Second argument: appendix title
\newcommand*{\appendixStart}[2][!-!-!]{%
  \sectionStart[#1]{#2}{Appendix}{#2}{\appendixname{}:~}}

\newcommand*{\appendixEnd}[1]{#1} % just a notational convenience

\newenvironment{histList}
  {\begin{list}
    {}
    {\setlength{\labelwidth}{108pt}% 1.5"
    \setlength{\leftmargin}{\labelwidth+\labelsep}
    \setlength{\rightmargin}{36pt}% 0.5"
    \setlength{\parsep}{0ex}
    \renewcommand{\makelabel}[1]{\textbf{##1}\hfill}
    }}
  {\end{list}}
\newcommand*{\histListBegin}{\begin{histList}}
\newcommand*{\histListEnd}{\end{histList}}
\newcommand{\histListItem}[1]{\item[#1]}

% Define hyphenation points:

\title{\cal\Huge\textitcorr{Movement~and~Meaning~Middleware}\\
\vspace{1ex}
\begin{center}\includegraphics{ChannelManager.eps}\end{center}}
\author{HPlus~Technologies~Ltd. and Simon~Fraser~University\\
Vancouver, British~Columbia, Canada}

\makeindex

%\listfiles
% End Preamble

\begin{document}

% Begin front matter

\maketitle

\insertpart{Contents}{\tableofcontents}
\insertpart{List~of~Figures}{\listoffigures}

\ProvidesFile{documentationHistory.tex}
\primaryStart[History]{Document~History}
\histListBegin

\histListItem{August 2014}{first version of this document}
\histListEnd

\primaryEnd{}

\ProvidesFile{frontispiece.tex}[1.0.0]
\primaryStart{Foreword}
\begin{quote}
\begin{small}
Copyright: \copyright{} 2014 by HPlus~Technologies~Ltd. and Simon~Fraser~University.
\\
All rights reserved. Redistribution and use in source and binary forms, with or without
modification, are permitted provided that the following conditions are met:\\
$\bullet$ Redistributions of source code must retain the above copyright notice, this list
of conditions and the following disclaimer.\\
$\bullet$ Redistributions in binary form must reproduce the above copyright notice, this
list of conditions and the following disclaimer in the documentation and/or other
materials provided with the distribution.\\
$\bullet$ Neither the name of the copyright holders nor the names of its contributors may
be used to endorse or promote products derived from this software without specific prior
written permission.\\
THIS SOFTWARE IS PROVIDED BY THE COPYRIGHT HOLDERS AND CONTRIBUTORS "AS IS" AND ANY
EXPRESS OR IMPLIED WARRANTIES, INCLUDING, BUT NOT LIMITED TO, THE IMPLIED WARRANTIES OF
MERCHANTABILITY AND FITNESS FOR A PARTICULAR PURPOSE ARE DISCLAIMED.
IN NO EVENT SHALL THE COPYRIGHT OWNER OR CONTRIBUTORS BE LIABLE FOR ANY DIRECT, INDIRECT,
INCIDENTAL, SPECIAL, EXEMPLARY, OR CONSEQUENTIAL DAMAGES (INCLUDING, BUT NOT LIMITED TO,
PROCUREMENT OF SUBSTITUTE GOODS OR SERVICES; LOSS OF USE, DATA, OR PROFITS; OR BUSINESS
INTERRUPTION) HOWEVER CAUSED AND ON ANY THEORY OF LIABILITY, WHETHER IN CONTRACT, STRICT
LIABILITY, OR TORT (INCLUDING NEGLIGENCE OR OTHERWISE) ARISING IN ANY WAY OUT OF THE USE
OF THIS SOFTWARE, EVEN IF ADVISED OF THE POSSIBILITY OF SUCH DAMAGE.

Adobe, the Adobe logo, Acrobat, the Acrobat logo, Distiller, Illustrator, Photoshop and
PageMaker are trademarks of
\companyReference{http://www.adobe.com}{Adobe Systems Incorporated}.

Apple, Applescript, Mac, the Mac logo, and Macintosh are trademarks of
\companyReference{http://www.apple.com}{Apple Incorporated}.

dvips(k) copyright \copyright{} 2013
\companyReference{http://www.radicaleye.com}{Radical Eye Software}.

TextWrangler copyright \copyright{}
\companyReference{http://www.barebones.com}{Bare Bones Software Incorporated}.

GPL Ghostscript copyright \copyright{} 2002
\companyReference{http://www.artifex.com}{Artifex Software Incorporated}.

GNU Make copyright \copyright{} 2006
\companyReference{http://www.gnu.org}{Free Software Foundation, Incorporated}.

cmake copyright \copyright{} 2000
\companyReference{http://www.kitware.com}{Kitware Incorporated}.

ACE copyright \copyright{} 1993--2009
\companyReference{http://www.cs.wustl.edu/\%7Eschmidt/ACE.html}{Douglas C. Schmidt}.

YARP copyright \copyright{} 2006
\companyReference{http://wiki.icub.org/yarpdoc/what_is_yarp.html}{RobotCub Consortium}.

JUCE copyright \copyright{} 2013
\companyReference{http://www.juce.com}{Raw Material Software Limited}.

OGDF copyright \copyright{} 2007
\companyReference{http://www.ogdf.net}{Open Graph Drawing Framework}.

MacTeX copyright \copyright{} 2005 \companyReference{https://tug.org/mactex}{MacTeX}.

Leap Motion copyright \copyright{} 2013
\companyReference{https://www.leapmotion.com}{Leap Motion, Incorporated}
\end{small}
\end{quote}
\vspace{\fill}
This document was created using \companyReference{https://tug.org/mactex}{MacTeX}-2014,
\companyReference{http://www.barebones.com}{TextWrangler} 4.5.10,
\companyReference{http://www.artifex.com}{GPL Ghostscript} 9.10,
\companyReference{http://www.gnu.org}{GNU Make version} 3.81,
\companyReference{http://www.tug.org/metapost.html}{MetaPost} 1.902 and
\companyReference{http://www.radicaleye.com}{dvips(k)} 5.994 on an Apple iMac.
\primaryEnd{}
\clearpage\pagenumbering{arabic}

% End of front matter
% Begin contents

\ProvidesFile{overview.tex}[v1.0.0]
\primaryStart{Overview}

\mplusm{} is a software system that acts as an intermediary between subsystems that
provide sensor data, such as accelerometers and motion capture cameras, and actuators
such as projectors and sound systems.
It provides mechanisms for reporting and interrogating the protocols used by the sensors
and actuators, as well as a standard architecture for creating services.

A \mplusm{} installation consists of a set of programs that interconnect using the
\companyReference{http://wiki.icub.org/yarpdoc/what_is_yarp.html}{YARP} networking
protocols, along with libraries that can be linked to applications to provide access to
the \mplusm{} features.

There are three main classes of programs in the set -- services, clients and utilities.
Clients use a formalized protocol to connect to services and utilities manage or monitor
the aggregate state of a \mplusm{} configuration.
There is a unique service, the \serviceNameR[Service~Registry]{ServiceRegistry}, that
maintains information on all the active services that are accessible to clients within a
\mplusm{} installation.
Each service can support multiple client connections, and the client functionality can be
embedded in command--line tools, GUI--based applications or headless background processes.

\primaryEnd{}


\ProvidesFile{utilities.tex}[v1.0.0]
\primaryStart{Utilities}

The utility programs that are provided with \mplusm{} provide access to the processes that
are running in the \mplusm{} installation.
Although command--line \yarp{} commands can also be used to manage the network
connections, it is recommended that the more specialized \mplusm{} tools be used, to avoid
inconsistencies.

\secondaryStart{GUI~Tools}

Currently there is one GUI--based tool, \utilityNameR[Channel~Manager]{ChannelManager},
which provides a view of the state of connections within an \mplusm{} installation, as
well as managing non--\mplusm{} \yarp{} network connections.

\tertiaryStart{\utilityNameP[Channel~Manager]{ChannelManager}}

The Channel~Manager application displays a single window view of the connections within a
\yarp{} network, with features designed to make management of an \mplusm{} installation
easier.\\

Simple \yarp{} network ports are shown as rectangles with a title consisting of the IP
address and port number of the port, and the \yarp{} name for the port as the body of the
rectangle, prefixed with `In' for input--only ports, `Out' for output--only ports and
`I/O' for general ports.\\

\mplusm{} services are shown as rectangles with a title consisting of the name provided by
the service, with the primary \yarp{} network connection as the first row in the body of
the rectangle, prefixed with `S' to indicate that it is a service connection.
Secondary \yarp{} network connections appear as rows below the primary connection,
prefixed with `In' for input--only connections and `Out' for output--only connections.
\mplusm{} Input~/~Output services do not have a visual appearance that is distinct from
other \mplusm{} services -- the connections that are allowed, however, are more
restricted.\\

\mplusm{} simple clients are shown as rectangles with a title consisting of the IP address
and port number of their connection to a service, with a row containing the \yarp{}
network connection prefixed with `C'.\\

Both \mplusm{} services and clients can have multiple secondary \yarp{} network ports.\\

\mplusm{} adapters are similar to \mplusm{} simple clients, except that they have
additional rows above the client--service \yarp{} network connection for the secondary
\yarp{} network connections, with prefixes of `In' for input--only connections and `Out'
for output--only connections.\\

Connections between ports are shown as lines with one of three thicknesses and one of
three colours.
The thinnest lines represent simple \yarp{} network connections, which have no explicit
behaviours.
The next thicker lines represent connections between Input~/~Output services; these
connections have specific behaviours.
The thickest lines represent connections between clients and services, which are not
modifiable by this tool.
TCP/IP connections, which are the default, are shown in teal, UDP connections are shown in
purple and other connections are shown in orange.
Note that the tool can only create TCP/IP or UDP connections.\\

To create a connection between two ports, click on the source port with the `Option' /
`Alt' key held down and either drag the mouse to the destination port or click on the
destination port.
If the `Shift' is held down at the same time as the `Option' / `Alt' key, the new
network connection will be UDP rather than TCP/IP.
While an `add' operation is active, the source port will be overlaid with a small filled
yellow circle.
To clear a pending `add' operation, either click on the source port a second time or click
somewhere other than a port.
Note that it is not possible to add a connection from an input--only port, from a client
port to a non--service port, or from an output--only port of an Input~/~Output service to
an input--only port of another Input~/~Output service if the ports do not have matching
protocols.\\

To remove a connection between two ports, click on the source port with the `Command' key
held down and then click on the destination port.
While a `remove' operation is active, the source port will be overlaid with a small
hollow yellow circle.
To clear a pending `remove' operation, either click on the source port a second time or
click on a port that is not connected to the source port or somewhere other than a port.
Note that is is not possible to remove a connection from a client to a service.\\

If one of the rectangular objects is clicked on with the `Control' key held down, an
information window will be presented.
If the title of the rectangle is clicked on, the information will be about the
object in general while, if a port is clicked on, the information will be about the
specific port.\\

Clicking on a rectangular object with no modifier key held down allows the user to drag
the object around in the window.
Note that connections cannot be selected by clicking on them.

\tertiaryEnd{\utilityNameE[Channel~Manager]{ChannelManager}}

\secondaryEnd{}

\secondaryStart{Command--Line~Utilities}

\mplusm{} includes a number of command--line tools that provide some of the functionality
of the GUI--based tool, \utilityNameR[Channel~Manager]{ChannelManager}.
As well, the native \yarp{} command--line tools to create and remove connections can be
used with \mplusm{}, but it is very easy to create a non--functional installation if care
is not taken.

\tertiaryStart{\utilityNameP{mpmClientList}}

			||TBD TBD TBD||

			||/TBD TBD TBD||

\tertiaryEnd{\utilityNameE{mpmClientList}}

\tertiaryStart{\utilityNameP{mpmPortLister}}

			||TBD TBD TBD||

			||/TBD TBD TBD||

\tertiaryEnd{\utilityNameE{mpmPortLister}}

\tertiaryStart{\utilityNameP{mpmRequestInfo}}

			||TBD TBD TBD||

			||/TBD TBD TBD||

\tertiaryEnd{\utilityNameE{mpmRequestInfo}}

\tertiaryStart{\utilityNameP{mpmServiceLister}}

			||TBD TBD TBD||

			||/TBD TBD TBD||

\tertiaryEnd{\utilityNameE{mpmServiceLister}}

\tertiaryStart{\utilityNameP{mpmVersion}}

The program \utilityNameR{mpmVersion} displays the version numbers for \mplusm{}, \yarp{}
and ACE, the low--level networking layer used by \mplusm{} and \yarp{}.
It has two command--line options:
\begin{itemize}
\item --j generate JSON--formatted output
\item --t generate output in tab--delimited form
\end{itemize}

If neither option is present, the output is written as a simple text.

\tertiaryEnd{\utilityNameE{mpmVersion}}

\secondaryEnd{}

\secondaryStart{Utility~Services~and~Clients}

			||TBD TBD TBD||

			||/TBD TBD TBD||

\tertiaryStart{\utilityNameP{mpmRequestCounterService}}

			||TBD TBD TBD||

			||/TBD TBD TBD||

\tertiaryEnd{\utilityNameE{mpmRequestCounterService}}

\tertiaryStart{\utilityNameP{mpmRequestCounterClient}}

			||TBD TBD TBD||

			||/TBD TBD TBD||

\tertiaryEnd{\utilityNameE{mpmRequestCounterClient}}

\secondaryEnd{}

\primaryEnd{}


%% section for each of the utilities

\ProvidesFile{services.tex}[v1.0.0]
\primaryStart[ServicesAndTheirProtocols]{Services~and~Their~Protocols}
An \mplusm{} installation consists of a number of service applications and their companion
client applications, communicating via \mplusm{} requests and responses on a \yarp{}
network.\\

There is one special service, the \serviceNameR[Service~Registry]{ServiceRegistry}, which
manages information about all other active services; all services register themselves
with the \serviceNameR[Service~Registry]{ServiceRegistry} so that client applications and
utilities can get information about the service.\\

All services and clients utilize a request / response protocol that is defined by the
\mplusm{} \compLang{C++} core classes.\\

The standard requests are in 4 groups:
\begin{itemize}
\item \textbf{\secondaryRef{BasicRequests}{Basic~Requests:}} requests that support the
fundamental \mplusm{} service mechanisms
\item \textbf{\secondaryRef{ServiceRegistryRequests}{Service~Registry~Requests:}} requests
that are specific to the \serviceNameR[Service~Registry]{ServiceRegistry}
\item \textbf{\secondaryRef{InputOutputRequests}{Input~/~Output~Requests:}} requests that
are specific to Input~/~Output services
\item \textbf{\secondaryRef{MiscellaneousRequests}{Miscellaneous~Requests:}} other
requests
\end{itemize}
Note that the \secondaryRef{BasicRequests}{Basic~Requests} are part of every service and
automatically supported in the base class of all services,
\begin{Large}\textbf{TBD--Base Service--TBD}\end{Large}.\\

When a client sends a request to a service, it can optionally request a response from the
service.
If no response is requested, the request is processed but no response is sent.
\secondaryStart[StandardServices]{Standard~Services}
Their are two standard services that are always part of an \mplusm{} installation.
\utilityNameR{mpmRequestCounterService} and its companion application
\utilityNameR{mpmRequestCounterClient} are described elsewhere, and the
\serviceNameR[Service~Registry]{ServiceRegistry} is described in the following section.
\tertiaryStart{mpmRegistryService (also known as the
\serviceNameD[Service~Registry]{ServiceRegistry})}
The \serviceNameX[Service~Registry]{ServiceRegistry} application is a background service
that is used to manage other services and their connections.
Its primary purpose is to serve as a repository of information on the active services in
an \mplusm{} installation.\\

It uses an internal database, as described in the
\appendixRef{InternalDatabaseStructure}{Internal~Database~Structure} appendix.
It responds to the requests in the
\secondaryRef{ServiceRegistryRequests}{Service~Registry~Requests} group.\\

Note that only one copy of the \serviceNameX[Service~Registry]{ServiceRegistry}
application can be running at a time in an \mplusm{} installation, due to its fixed
\yarp{} network port.
\tertiaryEnd{\serviceNameE[Service~Registry]{mpmRegistryService}}
\secondaryEnd{}
\secondaryStart[BasicRequests]{Basic~Requests}
These requests are implemented in all \mplusm{} services.
They constitute the fundamental mechanism that is used by the
\serviceNameR[Service~Registry]{ServiceRegistry} to identify each active service.
The base class for request handlers is
\begin{Large}\textbf{TBD--Base Request Handler--TBD}\end{Large}.
\tertiaryStart{\requestsNameD{Basic}{Basic}{channels}}
The \requestsNameX{Basic}{Basic}{channels} request returns a list of the secondary input
channels and a list of the secondary output channels for its service.\\

It is used by the \utilityNameR{mpmServiceLister} and
\utilityNameR[Channel~Manager]{ChannelManager} utilities to determine the active \yarp{}
network connections for each service.
\tertiaryEnd{\requestsNameE{Basic}{Basic}{channels}}
\tertiaryStart{\requestsNameD{Basic}{Basic}{clients}, alias:\ %
\requestsNameD{Basic}{Basic}{c}}
The \requestsNameX{Basic}{Basic}{clients} request returns a list of the active clients of
its service, if the service has \yarp{} network connections with persistent state.\\

It is used by the \utilityNameR{mpmClientList} utility to display a list of the clients
of services that have \yarp{} network connections with state.
\tertiaryEnd{\requestsNameE{Basic}{Basic}{clients}}
\tertiaryStart{\requestsNameD{Basic}{Basic}{detach}}
The \requestsNameX{Basic}{Basic}{detach} request is used by clients of a service to
indicate that they are no longer actively communicating with the service.\\

It is used by all client applications and adapters, such as
\examplesNameR{Clients}{mpmEchoClient},
\examplesNameR{Adapters}{mpmRandomNumberAdapter},\\
\examplesNameR{Clients}{mpmRandomNumberClient}, \utilityNameR{mpmRequestCounterClient},
\examplesNameR{Adapters}{mpmRunningSumAdapter},
\examplesNameR{Adapters}{mpmRunningSumAltAdapter} and
\examplesNameR{Clients}{mpmRunningSumClient}, to cleanly disconnect from their
corresponding service.
\tertiaryEnd{\requestsNameE{Basic}{Basic}{detach}}
\tertiaryStart{\requestsNameD{Basic}{Basic}{info}, alias:\ %
\requestsNameD{Basic}{Basic}{i}}
The \requestsNameX{Basic}{Basic}{info} request returns details about a single request
handled by a service.
The request details include the standard name of the request, the version number of the
request handler for the request, search keywords that can be used when matching requests,
a description of the request as well as representations of the expected arguments for the
request and output of the request.\\

It is used by the \utilityNameR{mpmRequestInfo} utility to gather information about a
request available from an active service.
\tertiaryEnd{\requestsNameE{Basic}{Basic}{info}}
\tertiaryStart{\requestsNameD{Basic}{Basic}{list}, alias:\ %
\requestsNameD{Basic}{Basic}{l}}
The \requestsNameX{Basic}{Basic}{list} request returns details about one or more requests
handled by a service.
The request details include the standard name of the request, the version number of the
request handler for the request, search keywords that can be used when matching requests,
a description of the request as well as representations of the expected arguments for the
request and output of the request.\\

It is used by the \serviceNameR[Service~Registry]{ServiceRegistry} and the
\utilityNameR{mpmRequestInfo} utility to gather information about the requests available
from each active service.
\tertiaryEnd{\requestsNameE{Basic}{Basic}{list}}
\tertiaryStart{\requestsNameD{Basic}{Basic}{name}, alias:\ %
\requestsNameD{Basic}{Basic}{n}}
The \requestsNameX{Basic}{Basic}{name} request returns details about its service, in the
form of the canonical name of the service, its description, its kind, the path to the
executable for the service and a description of the requests for the service.\\

It is used by the \serviceNameR[Service~Registry]{ServiceRegistry} and the
\utilityNameR{mpmServiceLister} and \utilityNameR[Channel~Manager]{ChannelManager}
utilities to collect basic information about each service.
\tertiaryEnd{\requestsNameE{Basic}{Basic}{name}}
\secondaryEnd{}
\secondaryStart[ServiceRegistryRequests]{Service~Registry~Requests}
The requests in this group are used exclusively by the
\serviceNameR[Service~Registry]{ServiceRegistry} application to manage its internal
database and to respond to information requests from client applications.
\tertiaryStart{\requestsNameD{Service~Registry}{ServiceRegistry}{associate}}
The \requestsNameX{Service~Registry}{ServiceRegistry}{associate} request is sent by
adapter applications to indicate that a particular \yarp{} network connection is part of
the same process as the client connection that sent the request.\\

It is used by the adapter applications \examplesNameR{Adapters}{mpmRandomNumberAdapter},
\examplesNameR{Adapters}{mpmRunningSumAdapter} and\\
\examplesNameR{Adapters}{mpmRunningSumAltAdapter} to indicate the input and output
\yarp{} network connections that are part of the application.
\tertiaryEnd{\requestsNameE{Service~Registry}{ServiceRegistry}{associate}}
\tertiaryStart{\requestsNameD{Service~Registry}{ServiceRegistry}{disassociate}}
The \requestsNameX{Service~Registry}{ServiceRegistry}{disassociate} request is sent by
adapter applications to indicate that a particular \yarp{} network connection is no
longer part of the same process as the client connection that sent the request.
This is normally done as part of the shutdown process for the application.\\

It is used by the adapter applications \examplesNameR{Adapters}{mpmRandomNumberAdapter},
\examplesNameR{Adapters}{mpmRunningSumAdapter} and\\
\examplesNameR{Adapters}{mpmRunningSumAltAdapter} to indicate that all the input and
output \yarp{} network connections that are part of the application are no longer active.
\tertiaryEnd{\requestsNameE{Service~Registry}{ServiceRegistry}{disassociate}}
\tertiaryStart{\requestsNameD{Service~Registry}{ServiceRegistry}{getAssociates}}
The \requestsNameX{Service~Registry}{ServiceRegistry}{getAssociates} request returns
a list of the input \yarp{} network connections and a list of the output \yarp{} network
connections for a given connection; if the requested port corresponds to a secondary
\yarp{} network connection, then the request returns the client connection that it is
associated with.\\

It is used by the \utilityNameR{mpmPortLister} and
\utilityNameR[Channel~Manager]{ChannelManager} utilities to identify relationships
between \yarp{} network connections.
\tertiaryEnd{\requestsNameE{Service~Registry}{ServiceRegistry}{getAssociates}}
\tertiaryStart{\requestsNameD{Service~Registry}{ServiceRegistry}{match}, alias:\ %
\requestsNameD{Service~Registry}{ServiceRegistry}{find}}
The \requestsNameX{Service~Registry}{ServiceRegistry}{match} request returns a list of
input \yarp{} network connections or service names for services that match the criteria
provided as an argument to the request.\\

It is used by all client and adapter applications to identify the service to which they
need to connect as well as by the \utilityNameR{mpmClientList},
\utilityNameR{mpmPortLister}, \utilityNameR{mpmRequestInfo},
\utilityNameR{mpmServiceLister} and \utilityNameR[Channel~Manager]{ChannelManager}
utilities to identify services in order to gather information about them.\\

For details on the syntax used with the criteria, see the 
\appendixRef{ServiceMatchSyntax}{Service~Match~Syntax} appendix.
\tertiaryEnd{\requestsNameE{Service~Registry}{ServiceRegistry}{match}}
\tertiaryStart{\requestsNameD{Service~Registry}{ServiceRegistry}{ping}}
The \requestsNameX{Service~Registry}{ServiceRegistry}{ping} request is sent by each
active service (except the \serviceNameR[Service~Registry]{ServiceRegistry}) to indicate
that it is still 'alive'.\\

Sending a \requestsNameX{Service~Registry}{ServiceRegistry}{ping} request results in a
sequence of requests being sent from the \serviceNameR[Service~Registry]{ServiceRegistry}
to the active service if the information on the service is no longer in the internal
database.
This will happen if the service is considered to be 'stale' -- it has not sent a
\requestsNameX{Service~Registry}{ServiceRegistry}{ping} request often enough to be
remembered.
The sequence of requests sent to the active service is described in the 
\appendixRef{RegistrationSequence}{Registration~Sequence} appendix.
\tertiaryEnd{\requestsNameE{Service~Registry}{ServiceRegistry}{ping}}
\tertiaryStart{\requestsNameD{Service~Registry}{ServiceRegistry}{register}, alias:\ %
\requestsNameD{Service~Registry}{ServiceRegistry}{remember}}
The \requestsNameX{Service~Registry}{ServiceRegistry}{register} request is sent by each
active service (except the \serviceNameR[Service~Registry]{ServiceRegistry}) when it
starts execution.\\

Sending a \requestsNameX{Service~Registry}{ServiceRegistry}{register} request results in a
sequence of requests being sent from the \serviceNameR[Service~Registry]{ServiceRegistry}
to the active service.
The sequence of requests sent to the active service is described in the 
\appendixRef{RegistrationSequence}{Registration~Sequence} appendix.
\tertiaryEnd{\requestsNameE{Service~Registry}{ServiceRegistry}{register}}
\tertiaryStart{\requestsNameD{Service~Registry}{ServiceRegistry}{unregister}, alias:\ %
\requestsNameD{Service~Registry}{ServiceRegistry}{forget}}
The \requestsNameX{Service~Registry}{ServiceRegistry}{unregister} request is sent by each
service (except the \serviceNameR[Service~Registry]{ServiceRegistry}) just before it stops
execution.\\

Upon receiving the \requestsNameX{Service~Registry}{ServiceRegistry}{unregister} request,
the \serviceNameR[Service~Registry]{ServiceRegistry} removes all entries from the internal
database that are related to the service being removed.
\tertiaryEnd{\requestsNameE{Service~Registry}{ServiceRegistry}{unregister}}
\secondaryEnd{}
\secondaryStart[InputOutputRequests]{Input~/~Output~Requests}

			||TBD TBD TBD||

			||/TBD TBD TBD||

\tertiaryStart{\requestsNameD{Input~/~Output}{InputOutput}{configure}}
The \requestsNameX{Input~/~Output}{InputOutput}{configure} request

			||TBD TBD TBD||

			||/TBD TBD TBD||

\tertiaryEnd{}
\tertiaryStart{\requestsNameD{Input~/~Output}{InputOutput}{restartStreams}}
The \requestsNameX{Input~/~Output}{InputOutput}{restartStreams} request

			||TBD TBD TBD||

			||/TBD TBD TBD||

\tertiaryEnd{\requestsNameE{Input~/~Output}{InputOutput}{restartStreams}}
\tertiaryStart{\requestsNameD{Input~/~Output}{InputOutput}{startStreams}}
The \requestsNameX{Input~/~Output}{InputOutput}{startStreams} request

			||TBD TBD TBD||

			||/TBD TBD TBD||

\tertiaryEnd{\requestsNameE{Input~/~Output}{InputOutput}{startStreams}}
\tertiaryStart{\requestsNameD{Input~/~Output}{InputOutput}{stopStreams}}
The \requestsNameX{Input~/~Output}{InputOutput}{stopStreams} request

			||TBD TBD TBD||

			||/TBD TBD TBD||

\tertiaryEnd{\requestsNameE{Input~/~Output}{InputOutput}{stopStreams}}
\secondaryEnd{}
\secondaryStart[MiscellaneousRequests]{Miscellaneous~Requests}

			||TBD TBD TBD||

			||/TBD TBD TBD||

\tertiaryStart{\requestsNameD{Miscellaneous}{Miscellaneous}{count}}
The \requestsNameX{Miscellaneous}{Miscellaneous}{count} request

			||TBD TBD TBD||

			||/TBD TBD TBD||

\tertiaryEnd{\requestsNameE{Miscellaneous}{Miscellaneous}{count}}
\tertiaryStart{\requestsNameD{Miscellaneous}{Miscellaneous}{echo}}
The \requestsNameX{Miscellaneous}{Miscellaneous}{echo} request

			||TBD TBD TBD||

			||/TBD TBD TBD||

\tertiaryEnd{\requestsNameE{Miscellaneous}{Miscellaneous}{echo}}
\tertiaryStart{\requestsNameD{Miscellaneous}{Miscellaneous}{reset}}
The \requestsNameX{Miscellaneous}{Miscellaneous}{reset} request

			||TBD TBD TBD||

			||/TBD TBD TBD||

\tertiaryEnd{\requestsNameE{Miscellaneous}{Miscellaneous}{reset}}
\tertiaryStart{\requestsNameD{Miscellaneous}{Miscellaneous}{stats}}
The \requestsNameX{Miscellaneous}{Miscellaneous}{stats} request

			||TBD TBD TBD||

			||/TBD TBD TBD||

\tertiaryEnd{\requestsNameE{Miscellaneous}{Miscellaneous}{stats}}
\secondaryEnd{}
\primaryEnd{}


%% section for each of the core services

\ProvidesFile{examples.tex}[v1.0.0]
\appendixStart{\textitcorr{Examples}}%

			\begin{Large}\textbf{TBD--TBD--TBD}\end{Large}.

\secondaryStart{Example~Services}

			\begin{Large}\textbf{TBD--TBD--TBD}\end{Large}.

\tertiaryStart{\examplesNameP{Services}{mpmEchoService}}

			\begin{Large}\textbf{TBD--TBD--TBD}\end{Large}.

\tertiaryEnd{\examplesNameE{Services}{mpmEchoService}}
\tertiaryStart{\examplesNameP{Services}{mpmRandomBurstStreamService}}

			\begin{Large}\textbf{TBD--TBD--TBD}\end{Large}.

\tertiaryEnd{\examplesNameE{Services}{mpmRandomBurstStreamService}}
\tertiaryStart{\examplesNameP{Services}{mpmRandomNumberService}}

			\begin{Large}\textbf{TBD--TBD--TBD}\end{Large}.

\tertiaryEnd{\examplesNameE{Services}{mpmRandomNumberService}}
\tertiaryStart{\examplesNameP{Services}{mpmRecordIntegersStreamService}}

			\begin{Large}\textbf{TBD--TBD--TBD}\end{Large}.

\tertiaryEnd{\examplesNameE{Services}{mpmRecordIntegersStreamService}}
\tertiaryStart{\examplesNameP{Services}{mpmRunningSumService}}

			\begin{Large}\textbf{TBD--TBD--TBD}\end{Large}.

\tertiaryEnd{\examplesNameE{Services}{mpmRunningSumService}}
\tertiaryStart{\examplesNameP{Services}{mpmTruncateFilterStreamService}}

			\begin{Large}\textbf{TBD--TBD--TBD}\end{Large}.

\tertiaryEnd{\examplesNameE{Services}{mpmTruncateFilterStreamService}}
\secondaryEnd{}
\secondaryStart{Example~Clients}

			\begin{Large}\textbf{TBD--TBD--TBD}\end{Large}.

\tertiaryStart{\examplesNameP{Clients}{mpmEchoClient}}

			\begin{Large}\textbf{TBD--TBD--TBD}\end{Large}.

\tertiaryEnd{\examplesNameE{Clients}{mpmEchoClient}}
\tertiaryStart{\examplesNameP{Clients}{mpmRandomNumberClient}}

			\begin{Large}\textbf{TBD--TBD--TBD}\end{Large}.

\tertiaryEnd{\examplesNameE{Clients}{mpmRandomNumberClient}}
\tertiaryStart{\examplesNameP{Clients}{mpmRunningSumClient}}

			\begin{Large}\textbf{TBD--TBD--TBD}\end{Large}.

\tertiaryEnd{\examplesNameE{Clients}{mpmRunningSumClient}}
\secondaryEnd{}
\secondaryStart{Example~Adapters}

			\begin{Large}\textbf{TBD--TBD--TBD}\end{Large}.

\tertiaryStart{\examplesNameP{Adapters}{mpmRandomNumberAdapter}}

			\begin{Large}\textbf{TBD--TBD--TBD}\end{Large}.

\tertiaryEnd{\examplesNameE{Adapters}{mpmRandomNumberAdapter}}
\tertiaryStart{\examplesNameP{Adapters}{mpmRunningSumAdapter}}

			\begin{Large}\textbf{TBD--TBD--TBD}\end{Large}.

\tertiaryEnd{\examplesNameE{Adapters}{mpmRunningSumAdapter}}
\tertiaryStart{\examplesNameP{Adapters}{mpmRunningSumAltAdapter}}

			\begin{Large}\textbf{TBD--TBD--TBD}\end{Large}.

\tertiaryEnd{\examplesNameE{Adapters}{mpmRunningSumAltAdapter}}
\secondaryEnd{}
\appendixEnd{}


%% section for each of the examples

\ProvidesFile{classes.tex}[v1.0.0]
\primaryStart[ClassesAndTheirInterfaces]{Classes~and~Their~Interfaces}
The \mplusm{} classes are organized into seven namespaces:
\begin{itemize}
\item \textbf{\mplusm{}::Common} classes that support all the \mplusm{} activities
\item \textbf{\mplusm{}::Example} classes that demonstrate features of \mplusm{}
\item \textbf{\mplusm{}::Parser} classes that are used by the
\requestsNameR{Service~Registry}{ServiceRegistry}{match} request of the
\serviceNameR[Service~Registry]{ServiceRegistry}
\item \textbf{\mplusm{}::Registry} classes that work with the internal database to manage
the \mplusm{} services
\item \textbf{\mplusm{}::RequestCounter} classes that support measuring the performance
of an \mplusm{} installation
\item \textbf{\mplusm{}::Test} classes that are used during unit--testing of the \mplusm{}
source code
\item \textbf{\mplusm{}::Utilities} classes that are used only with the \mplusm{}
utility programs
\end{itemize}
There are 38 classes or structures in the \textbf{\mplusm{}::Common} namespace, 27 in
\textbf{\mplusm{}::Example}, 8 in \textbf{\mplusm{}::Parser}, 14 in
\textbf{\mplusm{}::Registry}, 6 in \textbf{\mplusm{}::RequestCounter}, 20 in
\textbf{\mplusm{}::Test} and 3 in \textbf{\mplusm{}::Utilities}.
The following sections describe the most critical of the classes in the
\textbf{\mplusm{}::Common} namespace -- many of the other classes, in all the namespaces,
derive from these classes. 
\secondaryStart{BaseClient}
Instances of derived classes of \secondaryRef{BaseClient}{BaseClient} are responsible for
mediating \mplusm{} communication with corresponding instances of derived classes of
\secondaryRef{BaseService}{BaseService}.\\

The bulk of the mechanisms needed to perform this are provided by the
\secondaryRef{BaseClient}{BaseClient} class itself, and the derived classes need only
provide service--specific routines, such as \asCode{getOneRandomNumber()},
which is a method of the \asCode{M+M::Example::RandomNumberClient} class.\\

The following methods are available to code that uses
\secondaryRef{BaseClient}{BaseClient} classes:
\begin{itemize}
\item \textbf{\asCode{FindMatchingService}} This static function is provided for the use
of utility programs that need to identify active services
\item \textbf{\asCode{addAssociatedChannel}} Adapter applications use this method to
inform the \serviceNameR[Service~Registry]{ServiceRegistry} about their input and output
\yarp{} network connections
\item \textbf{\asCode{connectToService}} This method is use by clients to establish a
connection to the endpoint of their companion service
\item \textbf{\asCode{disconnectFromService}} This method is used by clients to release
the connection to their companion service
\item \textbf{\asCode{findService}} This method is used by clients to locate the channel
to be used for communication with their companion service
\item \textbf{\asCode{reconnectIfDisconnected}} This method is used by clients to
re--establish a dropped connection with their companion service
\item \textbf{\asCode{removeAssociatedChannels}} Adapter applications use this method to
inform the \serviceNameR[Service~Registry]{ServiceRegistry} that their input and output
\yarp{} network connections are no longer available
\item \textbf{\asCode{send}} This method is used by clients to send a request to their
companion service
\end{itemize}
\secondaryEnd{}
\secondaryStart{BaseContext}
Instances of derived classes of \secondaryRef{BaseService}{BaseService} that have \yarp{}
network connections with persistent state use instances of
derived classes of \secondaryRef{BaseContext}{BaseContext} to track the information that
is retained from each request for the following request.
The \secondaryRef{BaseContext}{BaseContext} class itself provides no functionality.
\secondaryEnd{}
\secondaryStart{BaseInputOutputService}
The \secondaryRef{BaseInputOutputService}{BaseInputOutputService} class is designed to
support easy access to multiple instances of the same service, each of which can be used
to provide an interface to an external device or to a simple flow--through operation.\\

Classes derived from \secondaryRef{BaseInputOutputService}{BaseInputOutputService} are
required to provide implementations of the following methods:
\begin{itemize}
\item \textbf{\asCode{configure}} This method is invoked when a
\requestsNameR{\inputOutput{}}{InputOutput}{configure} request is sent to the service; it
is responsible for setting attributes of the input and output streams of the service
\item \textbf{\asCode{restartStreams}} This method is invoked when a
\requestsNameR{\inputOutput{}}{InputOutput}{restartStreams} request is sent to the
service; it is responsible for stopping and then starting the input and output streams of
the service
\item \textbf{\asCode{setUpStreamDescriptions}} This method is invoked when the service is
started and it is responsible for providing a list of port names and protocols for use by
the current instance of the service
\item \textbf{\asCode{startStreams}} This method is invoked when a
\requestsNameR{\inputOutput{}}{InputOutput}{startStreams} request is sent to the service;
it is responsible for starting the input and output streams of the service
\item \textbf{\asCode{stopStreams}} This method is invoked when a
\requestsNameR{\inputOutput{}}{InputOutput}{stopStreams} request is sent to the service;
it is responsible for stopping the input and output streams of the service
\end{itemize}
The \asCode{configure}, \asCode{restartStreams}, \asCode{startStreams} and
\asCode{stopStreams} methods can also be invoked by code that instantiates the service.
\secondaryEnd{}
\secondaryStart{BaseMatcher}
The \secondaryRef{BaseMatcher}{BaseMatcher} class represents a `handle' to a top--down
pattern matcher used by the \serviceNameR[Service~Registry]{ServiceRegistry} to convert
the arguments of \requestsNameR{Service~Registry}{ServiceRegistry}{match} requests into
\asCode{SQL} operations.
The `outermost' matcher that is parsed from the original request is used to generate the
\asCode{SQL} -- the \secondaryRef{BaseMatcher}{BaseMatcher} class acts as a placeholder.
\secondaryEnd{}
\newpage
\secondaryStart{BaseRequestHandler}
The \secondaryRef{BaseRequestHandler}{BaseRequestHandler} class is designed to provide a
consistent request handling mechanism for \mplusm{} services.\\

Classes derived from \secondaryRef{BaseRequestHandler}{BaseRequestHandler} are required to
provide implementations of the following methods:
\begin{itemize}
\item \textbf{\asCode{fillInAliases}} This method is invoked when a service is registering
its request handlers; it is responsible for providing a list of alternative names for the
request
\item \textbf{\asCode{fillInDescription}} This method is invoked when a service is
responding to an \requestsNameR{Basic}{Basic}{info} or a
\requestsNameR{Basic}{Basic}{list} request; it is responsible for providing a detailed
description of the request
\item \textbf{\asCode{processRequest}} This method is invoked with the request is seen by
the service and it is responsible for performing the operations that correspond to the
request
\end{itemize}
\secondaryEnd{}
\secondaryStart{BaseService}
Instances of derived classes of \secondaryRef{BaseService}{BaseService} are responsible
for responding to requests from corresponding instances of derived classes of
\secondaryRef{BaseClient}{BaseClient}.
As well, they can respond on their secondary \yarp{} network connections to requests or
process data from the secondary connections.

Requests are mainly handled by instances of derived classes of the
\secondaryRef{BaseRequestHandler}{BaseRequestHandler} class that are owned by the service
instance; the \secondaryRef{BaseService}{BaseService} class manages the standard request
handlers.\\

Classes derived from \secondaryRef{BaseService}{BaseService} should implement the
following methods:
\begin{itemize}
\item \textbf{\asCode{attachRequestHandlers}} This method should be called in the
constructor for the service class and should register handlers for its custom requests
\item \textbf{\asCode{detachRequestHandlers}} This method should be called in the
destructor for the service class and should unregister handlers for its custom requests
\item \textbf{\asCode{start}} This method is invoked by the service application and should
invoke the same method of its parent class; it should set up any special conditions that
are needed by the service
\item \textbf{\asCode{stop}} This method is invoked by the service application and should
invoke the same method of its parent class; it should release any resources used by the
service
\end{itemize}
The following methods are available to code that uses
\secondaryRef{BaseService}{BaseService} classes:
\begin{itemize}
\item \textbf{\asCode{getEndpoint}} This method provides access to the
\secondaryRef{Endpoint}{Endpoint} that is provided for the service and is useful for
establishing a connection to an `embedded' service
\item \textbf{\asCode{getStatistics}} This method returns the call statistics for the
service
\end{itemize}
The following convenience methods are available to derived classes of
\secondaryRef{BaseService}{BaseService}:
\begin{itemize}
\item \textbf{\asCode{addContext}} This method is used to record the data that holds the
persistent state for a connection
\item \textbf{\asCode{clearContexts}} This method is used to remove all the persistent
state for connections
\item \textbf{\asCode{findContext}} This method is used to retrieve the data that holds
the persistent state of a connection
\item \textbf{\asCode{registerRequestHandler}} This method is used to register the
object that will handle a request
\item \textbf{\asCode{removeContext}} This method is used to remove the persistent state
of a connection
\item \textbf{\asCode{setDefaultRequestHandler}} This method is used to register the
object that will handle unknown requests
\item \textbf{\asCode{unregisterRequestHandler}} This method is used to unregister a
request handler
\end{itemize}
\secondaryEnd{}
\secondaryStart{Endpoint}
Instances of the \secondaryRef{Endpoint}{Endpoint} class manage a \yarp{} network
connection for a service.
They are used to simplify the communication mechanisms that each service must provide and
to standardize on the mid--level protocols that \mplusm{} services and clients share.
\secondaryEnd{}
\secondaryStart{ServiceRequest}
Instances of the \secondaryRef{ServiceRequest}{ServiceRequest} class are used to represent
requests in the mid--level protocols of \mplusm{} services in a uniform format.
\secondaryEnd{}
\secondaryStart{ServiceResponse}
Instances of the \secondaryRef{ServiceRequest}{ServiceRequest} class are used to represent
responses in the mid--level protocols of \mplusm{} services in a uniform format.
\secondaryEnd{}
\primaryEnd{}


%% section for each of the core classes

\appendix

\ProvidesFile{internalDb.tex}[v1.0.0]
\appendixStart[Internal~Database~Structure]{\textitcorr{Internal~Database~Structure}}%

			||TBD TBD TBD||

			||/TBD TBD TBD||

\appendixEnd{}


\ProvidesFile{registrationSequence.tex}[v1.0.0]
\appendixStart[Registration~Sequence]{\textitcorr{Registration~Sequence}}%

			||TBD TBD TBD||

			||/TBD TBD TBD||

\appendixEnd{}


% End of contents

\insertpart{Index}{\printindex}
\end{document}
\end
