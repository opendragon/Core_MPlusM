\ProvidesFile{utilities.tex}[v1.0.0]
\primaryStart{Utilities}

The utility programs that are provided with \mplusm{} provide access to the processes that
are running in the \mplusm{} installation.
Although command--line \yarp{} commands can also be used to manage the network
connections, it is recommended that the more specialized \mplusm{} tools be used, to avoid
inconsistencies.

\secondaryStart{GUI~Tools}

Currently there is one GUI--based tool, \utilityNameR[Channel~Manager]{ChannelManager},
which provides a view of the state of connections within an \mplusm{} installation, as
well as managing non--\mplusm{} \yarp{} network connections.

\tertiaryStart{\utilityNameP[Channel~Manager]{ChannelManager}}

The Channel~Manager application displays a single window view of the connections within a
\yarp{} network, with features designed to make management of an \mplusm{} installation
easier.\\

Simple \yarp{} network ports are shown as rectangles with a title consisting of the IP
address and port number of the port, and the \yarp{} name for the port as the body of the
rectangle, prefixed with `In' for input--only ports, `Out' for output--only ports and
`I/O' for general ports.\\

\mplusm{} services are shown as rectangles with a title consisting of the name provided by
the service, with the primary \yarp{} network connection as the first row in the body of
the rectangle, prefixed with `S' to indicate that it is a service connection.
Secondary \yarp{} network connections appear as rows below the primary connection,
prefixed with `In' for input--only connections and `Out' for output--only connections.
\mplusm{} Input~/~Output services do not have a visual appearance that is distinct from
other \mplusm{} services -- the connections that are allowed, however, are more
restricted.\\

\mplusm{} simple clients are shown as rectangles with a title consisting of the IP address
and port number of their connection to a service, with a row containing the \yarp{}
network connection prefixed with `C'.\\

Both \mplusm{} services and clients can have multiple secondary \yarp{} network ports.\\

\mplusm{} adapters are similar to \mplusm{} simple clients, except that they have
additional rows above the client--service \yarp{} network connection for the secondary
\yarp{} network connections, with prefixes of `In' for input--only connections and `Out'
for output--only connections.\\

Connections between ports are shown as lines with one of three thicknesses and one of
three colours.
The thinnest lines represent simple \yarp{} network connections, which have no explicit
behaviours.
The next thicker lines represent connections between Input~/~Output services; these
connections have specific behaviours.
The thickest lines represent connections between clients and services, which are not
modifiable by this tool.
TCP/IP connections, which are the default, are shown in teal, UDP connections are shown in
purple and other connections are shown in orange.
Note that the tool can only create TCP/IP or UDP connections.\\

To create a connection between two ports, click on the source port with the `Option' /
`Alt' key held down and either drag the mouse to the destination port or click on the
destination port.
If the `Shift' is held down at the same time as the `Option' / `Alt' key, the new
network connection will be UDP rather than TCP/IP.
While an `add' operation is active, the source port will be overlaid with a small filled
yellow circle.
To clear a pending `add' operation, either click on the source port a second time or click
somewhere other than a port.
Note that it is not possible to add a connection from an input--only port, from a client
port to a non--service port, or from an output--only port of an Input~/~Output service to
an input--only port of another Input~/~Output service if the ports do not have matching
protocols.\\

To remove a connection between two ports, click on the source port with the `Command' key
held down and then click on the destination port.
While a `remove' operation is active, the source port will be overlaid with a small
hollow yellow circle.
To clear a pending `remove' operation, either click on the source port a second time or
click on a port that is not connected to the source port or somewhere other than a port.
Note that is is not possible to remove a connection from a client to a service.\\

If one of the rectangular objects is clicked on with the `Control' key held down, an
information window will be presented.
If the title of the rectangle is clicked on, the information will be about the
object in general while, if a port is clicked on, the information will be about the
specific port.\\

Clicking on a rectangular object with no modifier key held down allows the user to drag
the object around in the window.
Note that connections cannot be selected by clicking on them.

\tertiaryEnd{\utilityNameE[Channel~Manager]{ChannelManager}}

\secondaryEnd{}

\secondaryStart{Command--Line~Utilities}

\mplusm{} includes a number of command--line tools that provide some of the functionality
of the GUI--based tool, \utilityNameR[Channel~Manager]{ChannelManager}.
As well, the native \yarp{} command--line tools to create and remove connections can be
used with \mplusm{}, but it is very easy to create a non--functional installation if care
is not taken.

\tertiaryStart{\utilityNameP{mpmClientList}}

			||TBD TBD TBD||

			||/TBD TBD TBD||

\tertiaryEnd{\utilityNameE{mpmClientList}}

\tertiaryStart{\utilityNameP{mpmPortLister}}

			||TBD TBD TBD||

			||/TBD TBD TBD||

\tertiaryEnd{\utilityNameE{mpmPortLister}}

\tertiaryStart{\utilityNameP{mpmRequestInfo}}

			||TBD TBD TBD||

			||/TBD TBD TBD||

\tertiaryEnd{\utilityNameE{mpmRequestInfo}}

\tertiaryStart{\utilityNameP{mpmServiceLister}}

			||TBD TBD TBD||

			||/TBD TBD TBD||

\tertiaryEnd{\utilityNameE{mpmServiceLister}}

\tertiaryStart{\utilityNameP{mpmVersion}}

The program \utilityNameR{mpmVersion} displays the version numbers for \mplusm{}, \yarp{}
and ACE, the low--level networking layer used by \mplusm{} and \yarp{}.
It has two command--line options:
\begin{itemize}
\item --j generate JSON--formatted output
\item --t generate output in tab--delimited form
\end{itemize}

If neither option is present, the output is written as a simple text.

\tertiaryEnd{\utilityNameE{mpmVersion}}

\secondaryEnd{}

\secondaryStart{Utility~Services~and~Clients}

			||TBD TBD TBD||

			||/TBD TBD TBD||

\tertiaryStart{\utilityNameP{mpmRequestCounterService}}

			||TBD TBD TBD||

			||/TBD TBD TBD||

\tertiaryEnd{\utilityNameE{mpmRequestCounterService}}

\tertiaryStart{\utilityNameP{mpmRequestCounterClient}}

			||TBD TBD TBD||

			||/TBD TBD TBD||

\tertiaryEnd{\utilityNameE{mpmRequestCounterClient}}

\secondaryEnd{}

\primaryEnd{}
