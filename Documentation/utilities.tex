\ProvidesFile{utilities.tex}[v1.0.0]
\primaryStart{Utilities}

The utility programs that are provided with \mplusm{} provide access to the processes that
are running in the \mplusm{} installation.
Although command--line \yarp{} commands can also be used to manage the network
connections, it is recommended that the more specialized \mplusm{} tools be used, to avoid
inconsistencies.

\secondaryStart{GUI~Tools}

Currently there is one GUI--based tool, \utilityNameR[Channel~Manager]{ChannelManager},
which provides a view of the state of connections within an \mplusm{} installation, as
well as managing non--\mplusm{} \yarp{} network connections.

\tertiaryStart{\utilityNameP[Channel~Manager]{ChannelManager}}

The Channel~Manager application displays a single window view of the connections within a
\yarp{} network, with features designed to make management of an \mplusm{} installation
easier.\\

Simple \yarp{} network ports are shown as rectangles with a title consisting of the IP
address and port number of the port, and the \yarp{} name for the port as the body of the
rectangle, prefixed with `In' for input--only ports, `Out' for output--only ports and
`I/O' for general ports.\\

\mplusm{} services are shown as rectangles with a title consisting of the name provided by
the service, with the primary \yarp{} network connection as the first row in the body of
the rectangle, prefixed with `S' to indicate that it is a service connection.
Secondary \yarp{} network connections appear as rows below the primary connection,
prefixed with `In' for input--only connections and `Out' for output--only connections.
\mplusm{} Input~/~Output services do not have a visual appearance that is distinct from
other \mplusm{} services -- the connections that are allowed, however, are more
restricted.\\

\mplusm{} simple clients are shown as rectangles with a title consisting of the IP address
and port number of their connection to a service, with a row containing the \yarp{}
network connection prefixed with `C'.\\

Both \mplusm{} services and clients can have multiple secondary \yarp{} network ports.\\

\mplusm{} adapters are similar to \mplusm{} simple clients, except that they have
additional rows above the client--service \yarp{} network connection for the secondary
\yarp{} network connections, with prefixes of `In' for input--only connections and `Out'
for output--only connections.\\

Connections between ports are shown as lines with one of three thicknesses and one of
three colours.
The thinnest lines represent simple \yarp{} network connections, which have no explicit
behaviours.
The next thicker lines represent connections between Input~/~Output services; these
connections have specific behaviours.
The thickest lines represent connections between clients and services, which are not
modifiable by this tool.
TCP/IP connections, which are the default, are shown in teal, UDP connections are shown in
purple and other connections are shown in orange.
Note that the tool can only create TCP/IP or UDP connections.\\

To create a connection between two ports, click on the source port with the `Option' /
`Alt' key held down and either drag the mouse to the destination port or click on the
destination port.
If the `Shift' is held down at the same time as the `Option' / `Alt' key, the new
network connection will be UDP rather than TCP/IP.
While an `add' operation is active, the source port will be overlaid with a small filled
yellow circle.
To clear a pending `add' operation, either click on the source port a second time or click
somewhere other than a port.
Note that it is not possible to add a connection from an input--only port, from a client
port to a non--service port, or from an output--only port of an Input~/~Output service to
an input--only port of another Input~/~Output service if the ports do not have matching
protocols.\\

To remove a connection between two ports, click on the source port with the `Command' key
held down and then click on the destination port.
While a `remove' operation is active, the source port will be overlaid with a small
hollow yellow circle.
To clear a pending `remove' operation, either click on the source port a second time or
click on a port that is not connected to the source port or somewhere other than a port.
Note that is is not possible to remove a connection from a client to a service.\\

If one of the rectangular objects is clicked on with the `Control' key held down, an
information window will be presented.
If the title of the rectangle is clicked on, the information will be about the
object in general while, if a port is clicked on, the information will be about the
specific port.\\

Clicking on a rectangular object with no modifier key held down allows the user to drag
the object around in the window.
Note that connections cannot be selected by clicking on them.

\tertiaryEnd{\utilityNameE[Channel~Manager]{ChannelManager}}

\secondaryEnd{}

\newpage
\secondaryStart{Command--Line~Utilities}

\mplusm{} includes a number of command--line tools that provide some of the functionality
of the GUI--based tool, \utilityNameR[Channel~Manager]{ChannelManager}.
As well, the native \yarp{} command--line tools to create and remove connections can be
used with \mplusm{}, but it is very easy to create a non--functional installation if care
is not taken.

\tertiaryStart{\utilityNameP{mpmClientList}}

The program \utilityNameR{mpmClientList} displays the clients for services that have
connections with state.
A service that has persistent state for its connections retains information from each
request for the following request.
An example service with persistent state is
\examplesNameR{Services}{mpmRunningSumService}, where the information that is kept is the
running sum for the connected client.
The program takes an optional argument for the name of the primary channel of the service;
if no argument is provided, all services are checked for connections with persistent
state.
It has two command--line options:
\begin{itemize}
\item \textbf{--j} generate JSON--formatted output
\item \textbf{--t} generate output in tab--delimited form
\end{itemize}

If neither option is present, the output is written as simple text.
The default output is:
\begin{quote}
Service: /\serviceName/example/runningSum\\
\settowidth{\utilLen}{Service: }%
Clients:\\
\hspace*{\utilLen}/\clientName/example/runningsum\textunderscore{}4896162
\end{quote}
or, if JSON--formatted output is requested:
\begin{quote}
[ \textbraceleft{} \dquote{}Service\dquote{}:
\dquote{}/\serviceName/example/runningSum\dquote{},
 \dquote{}Client\dquote{}:
 \dquote{}/\clientName/example/runningsum\textunderscore{}4896162\dquote{}
 \textbraceright{} ]
\end{quote}
or, if tab--delimited output is requested:
\begin{quote}
/\serviceName/example/runningSum\pseudotab{}%
/\clientName/example/runningsum\textunderscore{}4896162
\end{quote}
Note that, if no clients are found, the JSON--formatted output would be:
\begin{quote}
[  ]
\end{quote}
while the default output would be:
\begin{quote}
No client connections found.
\end{quote}
and the tab--delimited output would be empty.

\tertiaryEnd{\utilityNameE{mpmClientList}}

\tertiaryStart{\utilityNameP{mpmPortLister}}

The program \utilityNameR{mpmPortLister} displays the active \yarp{} ports and \mplusm{}
entities.
For each \yarp{} port, its role in the \mplusm{} installation is shown as well as any
incoming and outgoing \yarp{} network connections.
The primary port for each active service is identified, as well as the primary port for
each adapter.
It has two command--line options:
\begin{itemize}
\item \textbf{--j} generate JSON--formatted output
\item \textbf{--t} generate output in tab--delimited form
\end{itemize}

If neither option is present, the output is written as simple text.
The default output is:
\begin{quote}
Ports:\\
\settowidth{\utilLen}{Por}%
/\textdollar{}ervice: Service registry port for \squote{}Registry\squote{}.\\
\hspace*{\utilLen}No active connections.\\
/\textdollar{}ervice/status: Standard port at 10.0.1.2:10008.\\
\hspace*{\utilLen}Output to /reader via UDP.\\
/reader: Standard port at 10.0.1.2:10003.\\
\hspace*{\utilLen}Input from /writer2 via TCP.\\
\hspace*{\utilLen}Input from /writer via TCP.\\
\hspace*{\utilLen}Input from /\textdollar{}ervice/status via UDP.\\
/writer: Standard port at 10.0.1.2:10002.\\
\hspace*{\utilLen}Output to /reader via TCP.\\
/writer2: Standard port at 10.0.1.2:10004.\\
\hspace*{\utilLen}Output to /reader via TCP.
\end{quote}
or, if JSON--formatted output is requested:
\begin{quote}
[ \textbraceleft{} \dquote{}PortName\dquote{}: \dquote{}/\textdollar{}ervice\dquote{},
\dquote{}PortClass\dquote{}: \dquote{}Service~registry~port~for
\squote{}Registry\squote{}\dquote{}, \\
\dquote{}Inputs\dquote{}: [  ], \dquote{}Outputs\dquote{}: [  ] \textbraceright{},
\textbraceleft{} \dquote{}PortName\dquote{}:
\dquote{}/\textdollar{}ervice/status\dquote{}, \dquote{}PortClass\dquote{}:\\
\dquote{}Standard~port~at~10.0.1.2:10008\dquote{}, \dquote{}Inputs\dquote{}: [  ],
\dquote{}Outputs\dquote{}: [ \textbraceleft{} \dquote{}Port\dquote{}:
\dquote{}/reader\dquote{}, \\
\dquote{}Mode\dquote{}: \dquote{}UDP\dquote{} \textbraceright{} ] \textbraceright{},
\textbraceleft{} \dquote{}PortName\dquote{}: \dquote{}/reader\dquote{},
\dquote{}PortClass\dquote{}: \dquote{}Standard~port~at~10.0.1.2:10003\dquote{},\\
\dquote{}Inputs\dquote{}: [ \textbraceleft{} \dquote{}Port\dquote{}:
\dquote{}/writer2\dquote{}, \dquote{}Mode\dquote{}: \dquote{}TCP\dquote{}
\textbraceright{}, \textbraceleft{} \dquote{}Port\dquote{}: \dquote{}/writer\dquote{},
\dquote{}Mode\dquote{}: \dquote{}TCP\dquote{} \textbraceright{},\\
\textbraceleft{} \dquote{}Port\dquote{}: \dquote{}/\textdollar{}ervice/status\dquote{},
\dquote{}Mode\dquote{}: \dquote{}UDP\dquote{} \textbraceright{} ],
\dquote{}Outputs\dquote{}: [  ] \textbraceright{}, \textbraceleft{}
\dquote{}PortName\dquote{}: \dquote{}/writer\dquote{},\\
\dquote{}PortClass\dquote{}: \dquote{}Standard~port~at~10.0.1.2:10002\dquote{},
\dquote{}Inputs\dquote{}: [  ], \dquote{}Outputs\dquote{}: [ \textbraceleft{}
\dquote{}Port\dquote{}: \dquote{}/reader\dquote{},\\
\dquote{}Mode\dquote{}: \dquote{}TCP\dquote{} \textbraceright{} ] \textbraceright{},
\textbraceleft{} \dquote{}PortName\dquote{}: \dquote{}/writer2\dquote{},
\dquote{}PortClass\dquote{}: \dquote{}Standard~port~at~10.0.1.2:10004\dquote{},\\
\dquote{}Inputs\dquote{}: [  ], \dquote{}Outputs\dquote{}: [ \textbraceleft{}
\dquote{}Port\dquote{}: \dquote{}/reader\dquote{}, \dquote{}Mode\dquote{}:
\dquote{}TCP\dquote{} \textbraceright{} ] \textbraceright{} ]
\end{quote}
or, if tab--delimited output is requested:
\begin{quote}
/\textdollar{}ervice\pseudotab{}Service~registry~port~for~\squote{}Registry\squote{}\\		
/\textdollar{}ervice/status\pseudotab{}Standard~port~at~10.0.1.2:10008\pseudotab{}/reader~UDP\\
/reader\pseudotab{}Standard~port~at~10.0.1.2:10003\pseudotab{}/writer2~TCP,
/writer~TCP, /\textdollar{}ervice/status~UDP\\
/writer\pseudotab{}Standard~port~at~10.0.1.2:10002\pseudotab{}/reader~TCP\\
/writer2\pseudotab{}Standard~port~at~10.0.1.2:10004\pseudotab{}/reader~TCP
\end{quote}
Note that, if no ports are found, the JSON--formatted output would be:
\begin{quote}
[  ]
\end{quote}
while the default output would be:
\begin{quote}
Ports:\\
\settowidth{\utilLen}{Por}%
\hspace*{\utilLen}No ports found.
\end{quote}
and the tab--delimited output would be empty.

\tertiaryEnd{\utilityNameE{mpmPortLister}}

\tertiaryStart{\utilityNameP{mpmRequestInfo}}

			||TBD TBD TBD||

			||/TBD TBD TBD||

\tertiaryEnd{\utilityNameE{mpmRequestInfo}}

\tertiaryStart{\utilityNameP{mpmServiceLister}}

The program \utilityNameR{mpmServiceLister} displays the active services in the \mplusm{}
installation.
It lists each service, along with the service description and requests, as well as the
path to the executable for the service and the \yarp{} network ports that the service
provides.
It has two command--line options:
\begin{itemize}
\item \textbf{--j} generate JSON--formatted output
\item \textbf{--t} generate output in tab--delimited form
\end{itemize}

If neither option is present, the output is written as simple text.
The default output is:
\begin{quote}

Services:\\
\\
%Service port:      /\textdollar{}ervice
%Service name:      Registry
%Service kind:      Registry
%Description:       The Service Registry service
%Requests:          associate - associate a channel with another channel
%                   disassociate - remove all associations for a channel
%                   getAssociates - return the associations of a channel
%                   match - return the channels for services matching the criteria provided
%                   ping - update the last-pinged information for a channel or record the information for a service on the given channel
%                   register - record the information for a service on the given channel
%                   unregister - remove the information for a service on the given channel
%Path:              /Users/\textellipsis{}/mpmRegistryService
%Secondary outputs: /\textdollar{}ervice/status{protocol=s}
%
%Service port:      /\serviceName/example/randomburststream\textunderscore{}5580745
%Service name:      RandomBurst
%Service kind:      Input
%Description:       The random burst input service
%Requests:          
%Path:              /Users/\textellipsis{}/mpmRandomBurstStreamService
%Secondary outputs: /example/randomburststream/output\textunderscore{}29511ad{protocol=d+}
%
%Service port:      /\serviceName/example/recordintegersstream\textunderscore{}10efc07
%Service name:      RecordIntegers
%Service kind:      Output
%Description:       The record integers output service
%Requests:          
%Path:              /Users/\textellipsis{}/mpmRecordIntegersStreamService
%Secondary inputs:  /example/recordintegersstream/input\textunderscore{}5090e26{protocol=i+}
%
%Service port:      /\serviceName/example/truncatefilterstream\textunderscore{}2ee58eb
%Service name:      TruncateFilter
%Service kind:      Filter
%Description:       The truncate filter stream service
%Requests:          
%Path:              /Users/\textellipsis{}/mpmTruncateFilterStreamService
%Secondary inputs:  /example/truncatefilterstream/input\textunderscore{}39b1d9f{protocol=d+}
%Secondary outputs: /example/truncatefilterstream/output\textunderscore{}6a7d189{protocol=i+}

%Service: /\serviceName/example/runningSum\\
%\settowidth{\utilLen}{Service: }%
%Clients:\\
%\hspace*{\utilLen}/\clientName/example/runningsum\textunderscore{}4896162

\end{quote}
or, if JSON--formatted output is requested:
\begin{quote}
[ \textbraceleft{} \dquote{}ServicePort\dquote{}: \dquote{}/\textdollar{}ervice\dquote{},
\dquote{}ServiceName\dquote{}: \dquote{}Registry\dquote{}, \dquote{}ServiceKind\dquote{}:
\dquote{}Registry\dquote{}, \dquote{}Description\dquote{}: \dquote{}The Service Registry
service\dquote{}, \dquote{}Requests\dquote{}: \dquote{}associate -- associate a channel
with another channel\textbackslash{}n\\
disassociate -- remove all associations for a channel\textbackslash{}ngetAssociates --
return the associations of a\\
channel\textbackslash{}nmatch -- return the channels for services matching the criteria
provided\textbackslash{}nping -- update the last--pinged information for a channel or
record the information for a service on the given channel\textbackslash{}nregister --
record the information for a service on the given channel\textbackslash{}nunregister --
remove the information for a service on the given channel\dquote{}, \dquote{}Path%
\dquote{}: \dquote{}/Users/\textellipsis{}/mpmRegistryService\dquote{}, \dquote{}%
SecondaryInputs\dquote{}: [  ], \dquote{}SecondaryOutputs\dquote{}:
[ \textbraceleft{} \dquote{}Name\dquote{}: \dquote{}/\textdollar{}ervice/status\dquote{},
\dquote{}Protocol\dquote{}: \dquote{}s\dquote{} \textbraceright{} ] \textbraceright{},\\
\textbraceleft{} \dquote{}ServicePort\dquote{}: \dquote{}/\serviceName/example/%
randomburststream\textunderscore{}5580745\dquote{}, \dquote{}ServiceName\dquote{}:
\dquote{}RandomBurst\dquote{}, \dquote{}ServiceKind\dquote{}: \dquote{}Input\dquote{},
\dquote{}Description\dquote{}: \dquote{}The random burst input service\dquote{},
\dquote{}Requests\dquote{}: \dquote{}\dquote{}, \dquote{}Path\dquote{}:
\dquote{}/Users/\textellipsis{}/mpmRandomBurstStreamService\dquote{},
\dquote{}SecondaryInputs\dquote{}: [  ], \dquote{}SecondaryOutputs\dquote{}: [
\textbraceleft{} \dquote{}Name\dquote{}: \dquote{}%
/example/randomburststream/output\textunderscore{}29511ad\dquote{},
\dquote{}Protocol\dquote{}: \dquote{}d+\dquote{} \textbraceright{} ] \textbraceright{},\\
\textbraceleft{} \dquote{}ServicePort\dquote{}:
\dquote{}/\serviceName/example/recordintegersstream\textunderscore{}10efc07\dquote{},
\dquote{}ServiceName\dquote{}: \dquote{}RecordIntegers\dquote{},
\dquote{}ServiceKind\dquote{}: \dquote{}Output\dquote{}, \dquote{}Description\dquote{}:
\dquote{}The record integers output service\dquote{}, \dquote{}Requests\dquote{}:
\dquote{}\dquote{}, \dquote{}Path\dquote{}: \dquote{}/Users/\textellipsis{}/%
mpmRecordIntegersStreamService\dquote{}, \dquote{}SecondaryInputs\dquote{}: [
\textbraceleft{} \dquote{}Name\dquote{}:\\
\dquote{}/example/recordintegersstream/input\textunderscore{}5090e26\dquote{},
\dquote{}Protocol\dquote{}: \dquote{}i+\dquote{}
\textbraceright{} ], \dquote{}SecondaryOutputs\dquote{}: [  ] \textbraceright{},\\
\textbraceleft{} \dquote{}ServicePort\dquote{}:
\dquote{}/\serviceName/example/truncatefilterstream\textunderscore{}2ee58eb\dquote{},
\dquote{}ServiceName\dquote{}: \dquote{}TruncateFilter\dquote{},
\dquote{}ServiceKind\dquote{}: \dquote{}Filter\dquote{}, \dquote{}Description\dquote{}:
\dquote{}The truncate filter stream service\dquote{}, \dquote{}Requests\dquote{}:
\dquote{}\dquote{}, \dquote{}Path\dquote{}: \dquote{}/Users/\textellipsis{}/%
mpmTruncateFilterStreamService\dquote{}, \dquote{}SecondaryInputs\dquote{}: [
\textbraceleft{} \dquote{}Name\dquote{}:\\
\dquote{}/example/truncatefilterstream/input\textunderscore{}39b1d9f\dquote{},
\dquote{}Protocol\dquote{}: \dquote{}d+\dquote{} \textbraceright{} ],
\dquote{}SecondaryOutputs\dquote{}: [ \textbraceleft{} \dquote{}Name\dquote{}:\\
\dquote{}/example/truncatefilterstream/output\textunderscore{}6a7d189\dquote{},
\dquote{}Protocol\dquote{}: \dquote{}i+\dquote{} \textbraceright{} ] \textbraceright{} ]
\end{quote}
or, if tab--delimited output is requested:
\begin{quote}
/\textdollar{}ervice\pseudotab{}Registry\pseudotab{}Registry\pseudotab{}The Service
Registry service\pseudotab{}associate -- associate a channel with another
channel\textbackslash{}ndisassociate -- remove all associations for a
channel\textbackslash{}ngetAssociates -- return the\\
associations of a channel\textbackslash{}nmatch -- return the channels for services
matching the criteria provided\textbackslash{}nping\\
-- update the last--pinged information for a channel or record the information for a
service on the given\\
channel\textbackslash{}nregister --record the information for a service on the given
channel\textbackslash{}nunregister -- remove the\\
information for a service on the given channel\pseudotab{}%
/Users/\textellipsis{}/mpmRegistryService\pseudotab{}%
\pseudotab{}\\
/\textdollar{}ervice/status\textbraceleft{}protocol=s\textbraceright{}\\

/\serviceName/example/randomburststream\textunderscore{}5580745\pseudotab{}%
RandomBurst\pseudotab{}Input\pseudotab{}The random burst input\\
service\pseudotab{}\pseudotab{}%
/Users/\textellipsis{}/mpmRandomBurstStreamService\pseudotab{}\pseudotab{}\\
/example/randomburststream/output\textunderscore{}29511ad\textbraceleft{}protocol=d+%
\textbraceright{}\\

/\serviceName/example/recordintegersstream\textunderscore{}10efc07\pseudotab{}%
RecordIntegers\pseudotab{}Output\pseudotab{}The record integers\\
output service\pseudotab{}\pseudotab{}/Users/\textellipsis{}%
/mpmRecordIntegersStreamService\pseudotab{}\\
/example/recordintegersstream/input\textunderscore{}5090e26\textbraceleft{}%
protocol=i+\textbraceright{}\\

/\serviceName/example/truncatefilterstream\textunderscore{}2ee58eb\pseudotab{}%
TruncateFilter\pseudotab{}Filter\pseudotab{}The truncate filter stream\\
service\pseudotab{}\pseudotab{}/Users/\textellipsis{}/mpmTruncateFilterStreamService%
\pseudotab{}\\
/example/truncatefilterstream/input\textunderscore{}39b1d9f\textbraceleft{}%
protocol=d+\textbraceright{}\pseudotab{}\\
/example/truncatefilterstream/output\textunderscore{}6a7d189\textbraceleft{}%
protocol=i+\textbraceright{}
\end{quote}

For clarity, the full paths to the executables have been shortened and blank lines have
been added; in the actual output the paths would be absolute paths and there would be no
blank lines between the output rows.
Note that, if no services are found, the output would be empty.

\tertiaryEnd{\utilityNameE{mpmServiceLister}}

\tertiaryStart{\utilityNameP{mpmVersion}}

The program \utilityNameR{mpmVersion} displays the version numbers for \mplusm{}, \yarp{}
and ACE, the low--level networking layer used by \mplusm{} and \yarp{}.
It has two command--line options:
\begin{itemize}
\item \textbf{--j} generate JSON--formatted output
\item \textbf{--t} generate output in tab--delimited form
\end{itemize}

If neither option is present, the output is written as simple text.
The default output is:
\begin{quote}
Movement And Meaning Version: 1.4.8, YARP Version: 2.3.61, ACE Version: 6.2.6
\end{quote}
or, if JSON--formatted output is requested:
\begin{quote}
\textbraceleft{} \dquote{}M+M\dquote{}: \dquote{}1.4.8\dquote{}, \dquote{}YARP\dquote{}:
 \dquote{}2.3.61\dquote{}, \dquote{}ACE\dquote{}: \dquote{}6.2.6\dquote{}
 \textbraceright{}
\end{quote}
or, if tab--delimited output is requested:
\begin{quote}
1.4.8\pseudotab{}2.3.61\pseudotab{}6.2.6
\end{quote}

\tertiaryEnd{\utilityNameE{mpmVersion}}

\secondaryEnd{}

\newpage

\secondaryStart{Utility~Services~and~Clients}

			||TBD TBD TBD||

			||/TBD TBD TBD||

\tertiaryStart{\utilityNameP{mpmRequestCounterService}}

			||TBD TBD TBD||

			||/TBD TBD TBD||

\tertiaryEnd{\utilityNameE{mpmRequestCounterService}}

\tertiaryStart{\utilityNameP{mpmRequestCounterClient}}

			||TBD TBD TBD||

			||/TBD TBD TBD||

\tertiaryEnd{\utilityNameE{mpmRequestCounterClient}}

\secondaryEnd{}

\primaryEnd{}
