\ProvidesFile{MpMtechnology.tex}[v1.0.0]
\primaryStart{The Technology of \mplusm}
\objDiagram{mpm_images/MpMmpmDiagram}{mpmDiagram}{\mplusm{} structure}
\condPage{}
Figure \objDiagramRef{mpmDiagram} shows the logical organization of an \mplusm{}
installation, with some of the `physical' elements as well.
The brown lines represent client\longDash{}service communication, the blue lines represent
communication with the \serviceNameR[\RS]{RegistryService} and the black lines represent
(some of) the \yarp{} communication paths.\\

All \mplusm{} programs utilize \yarp{} to facilitate communication \longDash{} it provides
one\longDash{}to\longDash{}many output and many\longDash{}to\longDash{}one input
mechanisms, as well as a network\longDash{}based name server.
These input and output mechanisms are implemented via `mini\longDash{}server' code that is
a fundamental component of \yarp.\\

The `\yarp{} network' in the figure represents the aggregated TCP/IP connections that
exist when \yarp{} is active \longDash{} \yarp{} itself does not use any special protocols
and can operate over a variety of physical networks.\\

What \mplusm{} provides is a standardized client\longDash{}service mechanism, a set of
naming conventions for \yarp{} ports and a centralized database that is used to locate
services within the \yarp{} network, based on attributes of the services.\\

The \yarp{} Name Server is used to obtain the physical network address of each \mplusm{}
channel, given the name of the channel.
Once the network address is known, all communication between entities in \mplusm{} is via
either TCP/IP or UDP packets, using \yarp{} low\longDash{}level mechanisms to manage the
connections.\\

Services perform a sequence of requests and responses with the
\serviceNameR[\RS]{RegistryService} when they start, in order to be accessible from other
\mplusm{} entities.
Once started, they can receive requests from client applications, data streamed via their
input channels, external sensors or generated algorithmically, and transmit data via their
output channels, external transducers or files.
Additionally, they will receive periodic requests from the
\serviceNameR[\RS]{RegistryService}, inquiring as to their `health' and availability.\\

The \emph{\MMMU} is an essential tool for \mplusm{} applications, as it provides a
GUI\longDash{}based view into the connections within an \mplusm{} environment.
It also periodically probes the running services, in order to maintain its visual
representation of the state of the services, as well as to facilitate sending requests to
the services.
\primaryEnd{}
