\ProvidesFile{VDSservice.tex}[v1.0.0]
\primaryStart[TheVDSIService]{The~\VDSI{}~Service}
The \asCode{mpmViconDataStreamInputService} application is an Input service,
generating a stream of information on the position and orientation of one or more bodies.
The application responds to the standard Input service requests and can be used as a
standalone data generator, without the need for a client connection.\\

The \emph{configure} request has two arguments -- a string value for the host name of the
Vicon DataStream server and an integer value for the port on the Vicon DataStream server
to connect to.
These values will be applied when the output stream is started or restarted.\\ 

The \emph{restartStreams} request stops and then starts the output stream.\\

The \emph{startStreams} request request initiates listening to the Vicon DataStream
server.
Once started, the service will send groups of subject data via the output \yarp{} network
connection.\\

The \emph{stopStreams} request stops the Vicon DataStream server listener, which stops the
output \yarp{} network connection.\\ 

Note that the application will exit if the \emph{Registry Service} is not running.\\

The application has two optional arguments -- host name for the Vicon device server and
the port for the Vicon device server.
The application also has four optional parameters:
\begin{itemize}
\item \textbf{-e:} specifies an alternative endpoint name to be used
\item \textbf{-p:} specifies the port number to be used, if a non--default port is desired
\item \textbf{-r:} report the service metrics when the application exits
\item \textbf{-t:} specifies the tag to be used as part of the service name
\end{itemize}
The tag is added to the standard name of the service, so that more than one copy of the
service can execute -- an \mplusm{} installation can support multiple copies of each
\inputOutput{} service, but the \emph{Channel Manager} application cannot display them
without a distinguishing `tag'.
If the tag is not specified, the standard name of the service will be used.
As well as the service name, the output stream name is modified if a tag is specified and
the default endpoint is being used.\\

If the application is running from a terminal, the following commands are available:
\begin{itemize}
\item \textbf{?:} display a list of the available commands.
\item \textbf{b:} start the output stream, sending subject data. 
\item \textbf{c:} configure the service by providing the host name and port for the Vicon
DataStream server. 
\item \textbf{e:} stop the output stream. 
\item \textbf{q:} quit the application. 
\item \textbf{r:} restart the output stream. 
\item \textbf{u:} reset the configuration so that it will be reprocessed when the output
stream is restarted.
\end{itemize}
\primaryEnd{}
