\ProvidesFile{overview.tex}[v1.0.0]
\primaryStart{Overview}

\mplusm{} is a software system that acts as an intermediary between subsystems that
provide sensor data, such as accelerometers and motion capture cameras, and actuators
such as projectors and sound systems.
It provides mechanisms for reporting and interrogating the protocols used by the sensors
and actuators, as well as a standard architecture for creating services.

A \mplusm{} installation consists of a set of programs that interconnect using the
\companyReference{http://wiki.icub.org/yarpdoc/what_is_yarp.html}{YARP} networking
protocols, along with libraries that can be linked to applications to provide access to
the \mplusm{} features.

There are three main classes of programs in the set -- services, clients and utilities.
Clients use a formalized protocol to connect to services and utilities manage or monitor
the aggregate state of a \mplusm{} configuration.
There is a unique service, the \serviceNameR[Service~Registry]{ServiceRegistry}, that
maintains information on all the active services that are accessible to clients within a
\mplusm{} installation.
Each service can support multiple client connections, and the client functionality can be
embedded in command--line tools, GUI--based applications or headless background processes.

\primaryEnd{}
